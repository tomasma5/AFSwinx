\chapter{Závěr}
\section{Budoucí vývoj}
Testování s uživateli dopadlo dobře a na základě toho bychom chtěli framework dále rozvíjet. Serverová část frameworku umožňuje vytvářet definice komponent. Tyto informace lze využít ke stavbě komponent na libovolné platformě. V další iteraci, bychom rádi přidali možnost využívat tyto definice na mobilní platformě a v dynamicky se rozvíjejícím frameworku AngularJS. S tím se bude pojit vyjmutí specifických modulů z části AFSwinx a jejich propagace do jiného projektu, který pak bude využit na těchto platformách. Z hlediska bezpečnosti lze zatím využít autorizace typu basic, v tomto ohledu bychom rádi projekt rozšířili o další možnosti například o oAuth. Framework nyní podporuje množinu validací, kterou budeme dále rozšiřovat. Budou provedeny další UX testy na základě, kterých může být upraven způsob jakým framework funguje.

Iterace a rozvoj frameworku by měl vyústit v plně nasaditelnou technologii, kterou bychom chtěli distribuovat v rámci AspectFaces. Tím docílíme toho, že vývojář při využití tohoto projektu definují definice pouze jednou a klienti je budou moci používat na různých platformách. Výhodou bude velmi rychlé prototypování a vytváření aktivních prvků, které budou reflektovat změny v serverové konfiguraci a změny v modelu, který reprezentují.
\section{Zhodnocení práce}
Cílem práce bylo navrhnout způsob, jakým lze aspektově generovat uživatelská rozhraní na platformě Java SE. Práce se zaměřuje především na tlusté klienty k serverovým aplikacím. Framework byl navržen, sestaven a otestován pomocí ukázkové aplikace. Použitelnost frameworku byla testována při testech použitelnosti s cílovou skupinou uživatelů.

Nejprve bylo potřeba zpracovat již existující řešení a zhodnotit jejich výhody a nevýhody. Již v této fázi bylo zřejmé, že pokud budou objekty dat centralizovány, tak bude zjednodušena jejich správa a možné úpravy. Způsob má i jednu nevýhodu, pokud si klient od serveru vyžádá data, tak je struktura, kterou obdrží závislá na datech. V případě že se mění struktura dat, musí se přizpůsobovat i klient. Tuto nevýhodu lze odstranit dynamickou inspekcí dat, která vytvoří definice, jež se využijí k sestavení komponenty a poté ji nastaví na základě obdržených dat. Toto nastavení je dynamické, klient se tedy nemusí přizpůsobovat, pokud server změní strukturu dat, neboť mu tuto skutečnost server sdělí ihned při generování definic.

Byl vytvořen návrh definic dat, pomocí kterých umí klient vytvořit formuláře a tabulky. Tyto definice jsou generovány na serverové straně s využitím frameworku AspectFaces a poté odeslány na klienta, kde jsou interpretovány. Klient umí na základě těchto definic vytvářet tabulky či formuláře, které sestaví podle určeného layoutu a nastaví všem aktivním prvkům validátory. Při obdržení dat umí tyto data správně vložit do komponent a následně datový objekt znovu sestavit pouze ze znalosti definice. Klientská strana předem nezná objekt, jenž obdrží a může tedy pružně reagovat na změny datových definic. Z hlediska bezpečnosti podporuje klient komunikaci pomocí HTTPS a autorizaci typu basic. Framework, je tedy rozdělen na klientskou a serverovou část. S tím, že klientovi stačí vložení závislosti na klientskou stranu a serveru na serverovou stranu a v případě využití AspectFaces i závislost na tento projekt. Ke generování dat se využívá knihovna třetí strany. Výhodou je její stabilnost a fakt, že disponuje velkou škálou možností, jak uživatelské rozhraní generovat. Framework zohledňuje uživatelské role a nabízí anotace, které lze využít při generování. Šablony byly upraveny tak, aby vytvářeli jednotnou definici použitelnou na více platformách. Definice je tedy platformě nezávislá. 

Při vývoji frameworku jsem se musel na problém zaměřit jak z hlediska funkčnosti, tak z hlediska použitelnosti, a zohlednit budoucí vývoj frameworku. Bylo potřeba navrhnout způsob, jakým se bude model generovat a následně přenášet do klientské části aplikace. V klientské části bylo potřeba model zpracovat a na jeho základě sestavit konkrétní komponenty, s kterými může klient dále pracovat, a v kterých bude zohledněna bezpečnost, layout a vnitřní reprezentace dat. Tyto vlastnosti byly v analytické části rozpracovány a v implementační části vytvořeny. Výsledný framework byl otestován s uživateli při testech použitelnosti, v kterých jsem se zaměřil na způsob jakým uživatelé framework využívají. Některé části pokrývají unit testy a také byla vytvořena ukázková serverová a klientská aplikace. Tato ukázková aplikace sloužila k alfa testování a do budoucnosti bude sloužit pro smoke testy. Framework byl úspěšně navržen, vytvořen a otestován.