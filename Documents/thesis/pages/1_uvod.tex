\chapter{Úvod}
Tato diplomová práce se zabývá analýzou, návrhem a otestování generování uživatelského rozhranní na základě aspektů, které by bylo využitelné v platformě Java SE. Práce se zaměřuje zejména ne generování rozhranní v tlustých klientech neboť trendem moderní doby je využívat webové API serverů, z kterých jsou získávána data. V první části práci jsou popsány problém, jenž jsou s tímto procesem spojeny. Druhá část analyzuji způsob, jakým lze proces zjednodušit a navrhuje framework, který by celý proces zjendodušil. Třetí část diplomové práce popisuje vlastní implementaci a celkovou architekturu frameworku. Poslední část se zabývá testování a vytvořením ukázkového projekty, v kterém je framework použit.

Práce obsahuje sezanm použitých zkratek viz. příloha A, instalační a uživatelskou příručku viz. příloha B, použití UML diagramy viz. příloha C, ukázky zdrojového kódu a XML souborů viz. příloha D a samozřejmě zdrojové kódy, které jsou přiloženy na CD. Obsah tohoto CD je v příloze D.
\section{Motivace}
Vytváření uživatelských rozhranní je součástí téměř každé aplikace. Obvykle jsou zobrazovány formuláře, do kterých se zadávají data, která mohou být následně zobrazována v tabulkách. Vývoj uživatelských rozhranní je časové náročná věc, která obvykle podléha testování jak funkčnosti tak použitelnosti. Kromě toho lze předpokládat, že se bude navržené rozhranní měnit. Tyto změny jsou obvykly iniciovány zákazníkem. V případě, že jsou získávána data ze serveru, tak musí klientská aplikace znát strukturu dat na jejihž základě vytváří komponenty. Pokud je struktura změněna, tak je potřeba upravit i klienta. Generování uživatelského rozhranní za běhu aplikace tyto problémy odstraní, neboť umožní klientovi dynamicky reagovat na změnu dat. Na uživatelské rozhranní lze nahlížet i z dalších aspektů, jako je rozvržení komponent, bezpečnost a validace. Tyto aspekty vyžadují další čas na vývoj aplikace. Pokud bychom vytvořili framework, který automatizuje zadané procesy a správně interpretuje uživatelské rozhranní, pak bychom ušetřili čas a snížili náklady potřebné na vývoj této aplikace. Toto téma mi přišlo velmi zajímavé a když mi bylo nabídnuto vytvořit koncept, který by výše uvedené věci automatizoval, tak jsem neváhal a zpracoval téma jako diplomovou práci.