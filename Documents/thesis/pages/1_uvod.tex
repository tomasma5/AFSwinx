\chapter{Úvod}
Tato diplomová práce se zabývá analýzou, návrhem a otestováním generovaného uživatelského rozhraní na základě aspektů, které by bylo využitelné na platformě Java SE. Práce se~zaměřuje zejména na generování rozhraní v tlustých klientech neboť trendem moderní doby je využívat webové API serverů, z kterých jsou získávána data. V první části práce jsou popsány problémy, jenž jsou s tímto procesem spojeny. Druhá část analyzuje způsob, jakým lze proces zjednodušit a navrhuje framework, který by tohoto cíle dosáhl. Třetí část diplomové práce popisuje vlastní implementaci a celkovou architekturu frameworku. Poslední část se zabývá testováním a vytvořením ukázkového projektu, v kterém je framework použit.

Práce obsahuje seznam použitých zkratek viz. příloha A, instalační a uživatelskou příručku viz. příloha B, použité UML diagramy a obrázky viz. příloha C, ukázky zdrojového kódu a XML souborů viz. příloha D a samozřejmě zdrojové kódy, které jsou přiloženy na~CD. Obsah tohoto CD je v příloze E.
\section{Motivace}
Vytváření uživatelských rozhraní je součástí téměř každé aplikace. Obvykle jsou zobrazovány formuláře, do kterých se zadávají data, která mohou být následně zobrazována v tabulkách. Vývoj uživatelských rozhraní je časové náročná věc, která obvykle podléhá testování jak funkčnosti, tak použitelnosti. Kromě toho lze předpokládat, že se bude navržené rozhraní měnit. Tyto změny jsou obvykle iniciovány zákazníkem. V případě, že jsou získávána data ze serveru, tak musí klientská aplikace znát strukturu dat, na jejichž základě vytváří komponenty. Pokud je struktura změněna, pak je potřeba upravit i klienta. Generování uživatelského rozhraní za běhu aplikace tyto problémy odstraní, neboť umožní klientovi dynamicky reagovat na změnu dat. Na uživatelské rozhraní lze nahlížet i z dalších aspektů, jako je rozvržení komponent, bezpečnost a validace. Tyto aspekty vyžadují další čas na vývoj aplikace. Pokud bychom vytvořili framework, který automatizuje zadané procesy a správně interpretuje uživatelské rozhraní, pak bychom ušetřili čas a snížili náklady potřebné na vývoj této aplikace. Toto téma mi přišlo velmi zajímavé, a když mi bylo nabídnuto vytvořit koncept, který by~výše uvedené věci automatizoval, tak jsem neváhal a zpracoval téma jako diplomovou práci.