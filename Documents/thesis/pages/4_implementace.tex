\chapter{Implementace} 
\section{Architektura}
V tomto frameworku rozlišujeme klientskou a serverou část. Serverová část generuje data pro klienta a tímto způsobem ovlivňuje ovládací prvky, které klientská část aplikace zobrazuje uživateli. Diagram nasazení na obrázku \ref{img:deploymentFrameworkDiagram} zachycuje použití frameworku. Serverová část frameworku je nasazena na klientovi a je schopná generovat definice formulářů s použitím frameworku AspectFaces \cite{aspectFaces}. Tyto definice jsou převedeny na model, který je možné upravit a odeslat klientovi. Aby byla serverová část plně funkční je potřeba nasadit aplikaci, v které je využívána na Java EE aplikační server. Nicméně v případě využití pouze staticky generovaných definic, lze aplikaci nasadit na libovolný aplikační server, který bude poskytovat klientům definice kompatibilní s definici poskytovanými při dynamickém generování. Specifikace formuláře, je poté zaslána na klienta, který ji interpretuje za použítí klientské částy zvané AFSwinx. Tato část využívá i serverovou část a to z důvodu kompatibilnosti objektů a jejich vlastností. Přidání do projektu lze provést tak, že se do adresáři s knihovnami vloží přeložený jar soubor, či se přidá projekt jako Maven závislost. V současné době není framework k dispozici v centrálním repositáři, je tedy potřeba stáhnout aktuální verzi a zkompilovat ji do lokálního repositáře.  
\subsection{Server}
Jak již bylo zmíňěno, tak server využívá ke generování serverou část frameworku nazvanou AFRest. Na obrázku \ref{img:serverSide} jsou zobrazeny třídy a balíčky, které tato část využívá. Jsou zde výčtové typy, které určují podporované komponenty a jejich vlastnosti, dále objekty zodpovědné za informace o volbě layoutu a objekty nesoucí informace o definicích, na základě kterých budou sestaveny formuláře či tabulky klientem a samozřejmě třídy zodpovědné za inspekci dat. Framework doplňuje do AspectFaces několik anotací, které lze využít při generování definic. Jsou to následující anotace:
\begin{enumerate}
\item @UIWidgeType - tato anotace určuje typ widgetu, který se použije do xml šablon, které se používají při generování definic je propagován jako proměnná s názvem widgetType
\item @UILayout - tato anotace definuje layout na dané proměnné. Lze specifikovat typ layoutu, jeho orientace a pozice popisu prvku. Do xml šablon jsou tyto hodnoty propagovány jako layout, layoutOrientation a labelPossition. 
\end{enumerate}
Výše zmíněné anotace akceptují pouze hodnoty z výčtových typů v balíčku common. V případě typu komponenty nebo-li widgetType přijímá anotace hodnoty ze třídy SupportedWidgets a v případě anotace určující layout lze vložit pouze hodnoty z výčtových typů LayoutDefinitions, LayoutOrientation a LabelPosition. Hlavní výhodou tohoto řešení, je typová kontrola a jistota, že klient obdrží od serveru pouze takové hodnoty, s kterými je schopný pracovat.

\subsection{Klient}
