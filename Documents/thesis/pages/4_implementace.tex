\chapter{Iimplementace} 
\section{Architektura}
V tomto frameworku rozlišujeme klientskou a serverou část. Serverová část generuje data pro klienta a tímto způsobem ovlivňuje ovládací prvky, které klientská část aplikace zobrazuje uživateli. Diagram nasazení na obrázku \ref{img:deploymentFrameworkDiagram} zachycuje použití frameworku. Serverová část frameworku je nasazena na klientovi a je schopná generovat definice formulářů s použitím frameworku AspectFaces \cite{aspectFaces}. Tyto definice jsou převedeny na model, který je možné upravit a odeslat klientovi. Aby byla serverová část plně funkční je potřeba nasadit aplikaci, v které je využívána na Java EE aplikační server. Nicméně v případě využití pouze staticky generovaných definic, lze aplikaci nasadit na libovolný aplikační server, který bude poskytovat klientům definice kompatibilní s definici poskytovanými při dynamickém generování. Specifikace formuláře, je poté zaslána na klienta, který ji interpretuje za použítí klientské částy zvané AFSwinx. Tato část využívá i serverovou část a to z důvodu kompatibilnosti objektů a jejich vlastností. Přidání do projektu lze provést tak, že se do adresáři s knihovnami vloží přeložený jar soubor, či se přidá projekt jako Maven závislost. V současné době není framework k dispozici v centrálním repositáři, je tedy potřeba stáhnout aktuální verzi a zkompilovat ji do lokálního repositáře.  
\subsection{Server}
Jak již bylo zmíňěno, tak server využívá 
\subsection{Klient}
