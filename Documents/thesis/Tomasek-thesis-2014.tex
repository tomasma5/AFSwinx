%% History:
% Pavel Tvrdik (26.12.2004)
%  + initial version for PhD Report
%
% Daniel Sykora (27.01.2005)
%
% Michal Valenta (3.12.2008)
% rada zmen ve formatovani (diky M. Duškovi, J. Holubovi a J. Žďárkovi)
% sjednoceni zdrojoveho kodu pro anglickou, ceskou, bakalarskou a diplomovou praci

% One-page layout: (proof-)reading on display
%%%% \documentclass[11pt,oneside,a4paper]{book}
% Two-page layout: final printing
\documentclass[11pt,twoside,a4paper]{book}   
%=-=-=-=-=-=-=-=-=-=-=-=--=%
% The user of this template may find useful to have an alternative to these 
% officially suggested packages:
\usepackage[czech, english]{babel}

\usepackage[T1]{fontenc} % pouzije EC fonty 
% pripadne pisete-li cesky, pak lze zkusit take:
% \usepackage[OT1]{fontenc} 
\usepackage[utf8]{inputenc}
%For code
\usepackage{listings}
\usepackage{upquote}
\lstset{
%language=JAVA,
%columns=fullflexible,
%showstringspaces=false
%frame=tb
language=JAVA,
basicstyle=\small\sffamily,
frame=tb,
columns=fullflexible,
showstringspaces=false
}
%=-=-=-=-=-=-=-=-=-=-=-=--=%
% In case of problems with PDF fonts, one may try to uncomment this line:
%\usepackage{lmodern}
%=-=-=-=-=-=-=-=-=-=-=-=--=%
%=-=-=-=-=-=-=-=-=-=-=-=--=%
% Depending on your particular TeX distribution and version of conversion tools 
% (dvips/dvipdf/ps2pdf), some (advanced | desperate) users may prefer to use 
% different settings.
% Please uncomment the following style and use your CSLaTeX (cslatex/pdfcslatex) 
% to process your work. Note however, this file is in UTF-8 and a conversion to 
% your native encoding may be required. Some settings below depend on babel 
% macros and should also be modified. See \selectlanguage \iflanguage.
%\usepackage{czech}  %%%%%\usepackage[T1]{czech} %%%%[IL2] [T1] [OT1]
%=-=-=-=-=-=-=-=-=-=-=-=--=%

%%%%%%%%%%%%%%%%%%%%%%%%%%%%%%%%%%%%%%%
% Styles required in your work follow %
%%%%%%%%%%%%%%%%%%%%%%%%%%%%%%%%%%%%%%%
\usepackage{graphicx}
%\usepackage{indentfirst} %1. odstavec jako v cestine.

\usepackage{k336_thesis_macros} % specialni makra pro formatovani DP a BP
 % muzete si vytvorit i sva vlastni v souboru k336_thesis_macros.sty
 % najdete  radu jednoduchych definic, ktere zde ani nejsou pouzity
 % napriklad: 
 % \newcommand{\bfig}{\begin{figure}\begin{center}}
 % \newcommand{\efig}{\end{center}\end{figure}}
 % umoznuje pouzit prikaz \bfig namisto \begin{figure}\begin{center} atd.


%%%%%%%%%%%%%%%%%%%%%%%%%%%%%%%%%%%%%
% Zvolte jednu z moznosti 
% Choose one of the following options
%%%%%%%%%%%%%%%%%%%%%%%%%%%%%%%%%%%%%
\newcommand\TypeOfWork{Diplomová práce} \typeout{Diplomova prace}
% \newcommand\TypeOfWork{Master's Thesis}   \typeout{Master's Thesis} 
%\newcommand\TypeOfWork{Bakalářská práce}  \typeout{Bakalarska prace}
% \newcommand\TypeOfWork{Bachelor's Project}  \typeout{Bachelor's Project}


%%%%%%%%%%%%%%%%%%%%%%%%%%%%%%%%%%%%%
% Zvolte jednu z moznosti 
% Choose one of the following options
%%%%%%%%%%%%%%%%%%%%%%%%%%%%%%%%%%%%%
% nabidky jsou z: http://www.fel.cvut.cz/cz/education/bk/prehled.html

%\newcommand\StudProgram{Elektrotechnika a informatika, dobíhající, Bakalářský}
%\newcommand\StudProgram{Elektrotechnika a informatika, dobíhající, Magisterský}
 %\newcommand\StudProgram{Elektrotechnika a informatika, strukturovaný, Bakalářský}
% \newcommand\StudProgram{Elektrotechnika a informatika, strukturovaný, Navazující magisterský}
\newcommand\StudProgram{Otevřená informatika, Magisterský}
% English study:
% \newcommand\StudProgram{Electrical Engineering and Information Technology}  % bachelor programe
% \newcommand\StudProgram{Electrical Engineering and Information Technology}  %master program


%%%%%%%%%%%%%%%%%%%%%%%%%%%%%%%%%%%%%
% Zvolte jednu z moznosti 
% Choose one of the following options
%%%%%%%%%%%%%%%%%%%%%%%%%%%%%%%%%%%%%
% nabidky jsou z: http://www.fel.cvut.cz/cz/education/bk/prehled.html

%\newcommand\StudBranch{Výpočetní technika}   % pro program EaI bak. (dobihajici i strukt.)
%\newcommand\StudBranch{Výpočetní technika}   % pro prgoram EaI mag. (dobihajici i strukt.)
\newcommand\StudBranch{Softwarové inženýrství}            %pro STM
%\newcommand\StudBranch{Web a multimedia}                  % pro STM
%\newcommand\StudBranch{Computer Engineering}              % bachelor programe
%\newcommand\StudBranch{Computer Science and Engineering}  % master programe


%%%%%%%%%%%%%%%%%%%%%%%%%%%%%%%%%%%%%%%%%%%%
% Vyplnte nazev prace, autora a vedouciho
% Set up Work Title, Author and Supervisor
%%%%%%%%%%%%%%%%%%%%%%%%%%%%%%%%%%%%%%%%%%%%

\newcommand\WorkTitle{ Aspektově orientovaný vývoj uživatelských rozhraní pro~Java~SE~aplikace}
\newcommand\FirstandFamilyName{Bc. Martin Tomášek}
\newcommand\Supervisor{Ing. Tomáš Černý M.S.C.S. }


% Pouzijete-li pdflatex, tak je prijemne, kdyz bude mit vase prace
% funkcni odkazy i v pdf formatu
\usepackage[
pdftitle={\WorkTitle},
pdfauthor={\FirstandFamilyName},
bookmarks=true,
colorlinks=true,
breaklinks=true,
urlcolor=red,
citecolor=blue,
linkcolor=blue,
unicode=true,
]
{hyperref}



% Extension posted by Petr Dlouhy in order for better sources reference (\cite{} command) especially in Czech.
% April 2010
% See comment over \thebibliography command for details.

\usepackage[square, numbers]{natbib}             % sazba pouzite literatury
%\usepackage{url}
%\DeclareUrlCommand\url{\def\UrlLeft{<}\def\UrlRight{>}\urlstyle{tt}}  %rm/sf/tt
%\renewcommand{\emph}[1]{\textsl{#1}}    % melo by byt kurziva nebo sklonene,
\let\oldUrl\url
\renewcommand\url[1]{<\texttt{\oldUrl{#1}}>}




\begin{document}

%%%%%%%%%%%%%%%%%%%%%%%%%%%%%%%%%%%%%
% Zvolte jednu z moznosti 
% Choose one of the following options
%%%%%%%%%%%%%%%%%%%%%%%%%%%%%%%%%%%%%
\selectlanguage{czech}
%\selectlanguage{english} 

% prikaz \typeout vypise vyse uvedena nastaveni v prikazovem okne
% pro pohodlne ladeni prace


\iflanguage{czech}{
	 \typeout{************************************************}
	 \typeout{Zvoleny jazyk: cestina}
	 \typeout{Typ prace: \TypeOfWork}
	 \typeout{Studijni program: \StudProgram}
	 \typeout{Obor: \StudBranch}
	 \typeout{Jmeno: \FirstandFamilyName}
	 \typeout{Nazev prace: \WorkTitle}
	 \typeout{Vedouci prace: \Supervisor}
	 \typeout{***************************************************}
	 \newcommand\Department{Katedra počítačové grafiky a interakce}
	 \newcommand\Faculty{Fakulta elektrotechnická}
	 \newcommand\University{České vysoké učení technické v Praze}
	 \newcommand\labelSupervisor{Vedoucí práce}
	 \newcommand\labelStudProgram{Studijní program}
	 \newcommand\labelStudBranch{Obor}
}{
	 \typeout{************************************************}
	 \typeout{Language: english}
	 \typeout{Type of Work: \TypeOfWork}
	 \typeout{Study Program: \StudProgram}
	 \typeout{Study Branch: \StudBranch}
	 \typeout{Author: \FirstandFamilyName}
	 \typeout{Title: \WorkTitle}
	 \typeout{Supervisor: \Supervisor}
	 \typeout{***************************************************}
	 \newcommand\Department{Department of Computer Science and Engineering}
	 \newcommand\Faculty{Faculty of Electrical Engineering}
	 \newcommand\University{Czech Technical University in Prague}
	 \newcommand\labelSupervisor{Supervisor}
	 \newcommand\labelStudProgram{Study Programme} 
	 \newcommand\labelStudBranch{Field of Study}
}




%%%%%%%%%%%%%%%%%%%%%%%%%%    Poznamky ke kompletaci prace
% Nasledujici pasaz uzavrenou v {} ve sve praci samozrejme 
% zakomentujte nebo odstrante. 
% Ve vysledne svazane praci bude nahrazena skutecnym 
% oficialnim zadanim vasi prace.

%%%%%%%%%%%%%%%%%%%%%%%%%%    Titulni stranka / Title page 

\coverpagestarts

%%%%%%%%%%%%%%%%%%%%%%%%%%%    Podekovani / Acknowledgements 

\acknowledgements
\noindent
Tímto bych rád poděkoval celé svojí rodině za podporu během studia. Dále bych rád poděkoval vedoucímu mé diplomové práce panu Ing. Tomáši Černému za ochotu, pomoc, čas, zkušenosti a příležitosti, které mi poskytoval během celého mého studia. 

%%%%%%%%%%%%%%%%%%%%%%%%%%%   Prohlaseni / Declaration 

\declaration{Ve~Strakonicích 28.\,12.\,2014}
%\declaration{In Kořenovice nad Bečvárkou on May 15, 2008}


%%%%%%%%%%%%%%%%%%%%%%%%%%%%    Abstract 
 
\abstractpage
Present-day applications use graphical user interface in order to interact with the user. We can find the input fields in each application together with the output fields and other components that are used to operate the application. To generate the forms that contain data that we want to insert, edit or view is time-consuming. In addition to generating the correct input fields, it is also necessary to consider the production of the user interface in terms of validation, distribution of components and security. In these cases, it would be appropriate that the parts of the interface were generated based on a model. In the case of client server types of applications, the model and the data can be provided by a server. This solution will provide a centralized data management and its definition and will allow the client to respond flexibly to any changes in the data model.

% Prace v cestine musi krome abstraktu v anglictine obsahovat i
% abstrakt v cestine.
\vglue60mm

\noindent{\Huge \textbf{Abstrakt}}
\vskip 2.75\baselineskip

\noindent
Současné aplikace využívají grafické uživatelské rozhraní k interakci s uživatelem. V každé aplikaci můžeme najít vstupní pole, výstupní pole a další komponenty, které slouží k ovládání aplikace. Generování formulářů, která obsahují data, jenž chceme vkládat, editovat nebo prohlížet je časově náročné. Kromě generování správných vstupních polí je také potřeba nahlížet na tvorbu uživatelského rozhraní z pohledu validací, rozložení komponent a bezpečnosti. V~těchto případech by bylo vhodné, aby se části rozhraní generovala na základě modelu. V~případě aplikací typu klient server může model a data poskytovat server. Toto řešení zajistí centralizovanou správu dat a jejich definicí a umožní klientovi pružně reagovat na změnu datového modelu.

%%%%%%%%%%%%%%%%%%%%%%%%%%%%%%%%  Obsah / Table of Contents 

\tableofcontents


%%%%%%%%%%%%%%%%%%%%%%%%%%%%%%%  Seznam obrazku / List of Figures 

\listoffigures


%%%%%%%%%%%%%%%%%%%%%%%%%%%%%%%  Seznam tabulek / List of Tables

\listoftables
\renewcommand\lstlistingname{Část zdrojového kódu}
\renewcommand\lstlistlistingname{Seznam částí zdrojových kódů}
\lstlistoflistings


%**************************************************************

\mainbodystarts
% horizontalní mezera mezi dvema odstavci
%\parskip=5pt
%11.12.2008 parskip + tolerance
\normalfont
\parskip=0.2\baselineskip plus 0.2\baselineskip minus 0.1\baselineskip

% Odsazeni prvniho radku odstavce resi class book (neaplikuje se na prvni 
% odstavce kapitol, sekci, podsekci atd.) Viz usepackage{indentfirst}.
% Chcete-li selektivne zamezit odsazeni 1. radku nektereho odstavce,
% pouzijte prikaz \noindent.

%**************************************************************

% Pro snadnejsi praci s vetsimi texty je rozumne tyto rozdelit
% do samostatnych souboru nejlepe dle kapitol a tyto potom vkladat
% pomoci prikazu \include{jmeno_souboru.tex} nebo \include{jmeno_souboru}.
% Napr.:
% \chapter{Úvod}
Tato diplomová práce se zabývá analýzou, návrhem a otestováním generovaného uživatelského rozhraní na základě aspektů, které by bylo využitelné na platformě Java SE. Práce se~zaměřuje zejména na generování rozhraní v tlustých klientech neboť trendem moderní doby je využívat webové API serverů, z kterých jsou získávána data. V první části práce jsou popsány problémy, jenž jsou s tímto procesem spojeny. Druhá část analyzuje způsob, jakým lze proces zjednodušit a navrhuje framework, který by tohoto cíle dosáhl. Třetí část diplomové práce popisuje vlastní implementaci a celkovou architekturu frameworku. Poslední část se zabývá testováním a vytvořením ukázkového projektu, v kterém je framework použit.

Práce obsahuje seznam použitých zkratek viz. příloha A, instalační a uživatelskou příručku viz. příloha B, použité UML diagramy a obrázky viz. příloha C, ukázky zdrojového kódu a XML souborů viz. příloha D a samozřejmě zdrojové kódy, které jsou přiloženy na~CD. Obsah tohoto CD je v příloze E.
\section{Motivace}
Vytváření uživatelských rozhraní je součástí téměř každé aplikace. Obvykle jsou zobrazovány formuláře, do kterých se zadávají data, která mohou být následně zobrazována v tabulkách. Vývoj uživatelských rozhraní je časové náročná věc, která obvykle podléhá testování jak funkčnosti, tak použitelnosti. Kromě toho lze předpokládat, že se bude navržené rozhraní měnit. Tyto změny jsou obvykle iniciovány zákazníkem. V případě, že jsou získávána data ze serveru, tak musí klientská aplikace znát strukturu dat, na jejichž základě vytváří komponenty. Pokud je struktura změněna, pak je potřeba upravit i klienta. Generování uživatelského rozhraní za běhu aplikace tyto problémy odstraní, neboť umožní klientovi dynamicky reagovat na změnu dat. Na uživatelské rozhraní lze nahlížet i z dalších aspektů, jako je rozvržení komponent, bezpečnost a validace. Tyto aspekty vyžadují další čas na vývoj aplikace. Pokud bychom vytvořili framework, který automatizuje zadané procesy a správně interpretuje uživatelské rozhraní, pak bychom ušetřili čas a snížili náklady potřebné na vývoj této aplikace. Toto téma mi přišlo velmi zajímavé, a když mi bylo nabídnuto vytvořit koncept, který by~výše uvedené věci automatizoval, tak jsem neváhal a zpracoval téma jako diplomovou práci.
% \include{2_teorie}
% atd...

%*****************************************************************************
\chapter{Úvod}
Tato diplomová práce se zabývá analýzou, návrhem a otestováním generovaného uživatelského rozhraní na základě aspektů, které by bylo využitelné na platformě Java SE. Práce se~zaměřuje zejména na generování rozhraní v tlustých klientech neboť trendem moderní doby je využívat webové API serverů, z kterých jsou získávána data. V první části práce jsou popsány problémy, jenž jsou s tímto procesem spojeny. Druhá část analyzuje způsob, jakým lze proces zjednodušit a navrhuje framework, který by tohoto cíle dosáhl. Třetí část diplomové práce popisuje vlastní implementaci a celkovou architekturu frameworku. Poslední část se zabývá testováním a vytvořením ukázkového projektu, v kterém je framework použit.

Práce obsahuje seznam použitých zkratek viz. příloha A, instalační a uživatelskou příručku viz. příloha B, použité UML diagramy a obrázky viz. příloha C, ukázky zdrojového kódu a XML souborů viz. příloha D a samozřejmě zdrojové kódy, které jsou přiloženy na~CD. Obsah tohoto CD je v příloze E.
\section{Motivace}
Vytváření uživatelských rozhraní je součástí téměř každé aplikace. Obvykle jsou zobrazovány formuláře, do kterých se zadávají data, která mohou být následně zobrazována v tabulkách. Vývoj uživatelských rozhraní je časové náročná věc, která obvykle podléhá testování jak funkčnosti, tak použitelnosti. Kromě toho lze předpokládat, že se bude navržené rozhraní měnit. Tyto změny jsou obvykle iniciovány zákazníkem. V případě, že jsou získávána data ze serveru, tak musí klientská aplikace znát strukturu dat, na jejichž základě vytváří komponenty. Pokud je struktura změněna, pak je potřeba upravit i klienta. Generování uživatelského rozhraní za běhu aplikace tyto problémy odstraní, neboť umožní klientovi dynamicky reagovat na změnu dat. Na uživatelské rozhraní lze nahlížet i z dalších aspektů, jako je rozvržení komponent, bezpečnost a validace. Tyto aspekty vyžadují další čas na vývoj aplikace. Pokud bychom vytvořili framework, který automatizuje zadané procesy a správně interpretuje uživatelské rozhraní, pak bychom ušetřili čas a snížili náklady potřebné na vývoj této aplikace. Toto téma mi přišlo velmi zajímavé, a když mi bylo nabídnuto vytvořit koncept, který by~výše uvedené věci automatizoval, tak jsem neváhal a zpracoval téma jako diplomovou práci.
\chapter{Popis problému a specifikace cíle}
\section{Popis problematiky}
Softwarové systémy jsou určeny k tomu, aby méně či více úspěšně poskytovali uživateli nástroj, který mu pomůže s řešením problémů. Systém tedy musí komunikovat s uživatelem. K tomuto účelu se využívá uživatelské rozhraní. Vývoj uživatelského rozhraní zabere přibližně 60% času, který je určen na vývoj konkrétního systému. Tento údaj se samozřejmě může lišit v závislosti na účelu a velikosti systému. Při tvorbě uživatelského rozhraní se obvykle zaměřujeme na použitelnost. Zde se zaměřujeme již na cílovou skupinu. Provádíme testy použitelnosti a na základě těchto a dalších testů jsme schopni určit, zdali je návrh použitelný či nikoliv. Důvodem tohoto testování je fakt, že obvykle systém stavíme pro uživatele ne obráceně. Z výše uvedených skutečností vyplívá, že je potřeba uživatelské rozhraní důkladně testovat, aby bylo pro cílovou skupinu správně použitelné. Bohužel, když se mluví o uživatelském rozhraní, tak se často zapomíná na to, že toto rozhraní se musí nejen vytvořit, ale také udržovat. Softwarový systém tráví většinu svého života v udržovacím režimu, kterému se říká support. V této fázi přichází na systém mnoho požadavků, které musí být proveditelné a to za přijatelné náklady. Nedílnou součástí jsou změny, které se týkají databázového modelu a obvykle tyto změny musí reflektovat UI. Podívejme se proto na systém z pohledu vývojáře. Systém pro něj musí být snadno udržovatelný, změny lehce proveditelné a bez větších dopadů na systém. V tomto případě by bylo vhodné reflektovat tyto změny v UI. 
\subsection{Typy uživatelských rozhraní}
Jak již bylo zmíněno, tak uživatelské rozhraní se testuje na základě typu aplikace a jejím použití. Je také důležité vzít v potaz zařízení, na kterém je aplikace provozována. Může se jednat o desktopovou, mobilní či serverou aplikaci. V každém z výše uvedených případů bude návrh uživatelského rohraní podíměn jinými faktory, které jsou specifické pro dané zařízení. Těmito faktory jsou způsoby, jakým se aplikace ovládá, prostředí, v kterém se uživatel právě nachází a účel, ke kterému je aplikace určena. Například aplikace na mobilních zařízení nemusí podporovat klávesové zkratky, ale mohla by podporovat gesta. Obdobně aplikace použitá na desktopu může počítat s použití myše, touchpadu, klávesnice - jak standardní tak dotykové či jiného externího zařízení. Je tedy zřejmé, že uživatelské rozhraní je kromě jeho účelu podmíněno i zařízením, na kterém je používáno.
\subsection{Získávání dat}
 
\chapter{Analýza}
\section{Funkční specifikace}
Framework \cite{framework} musí uživateli umožňovat vytvářet a dále pracovat s vytvořenými komponentami. Komponenty budou procházet určitým životním cyklem. Kromě samotných komponent musí framework poskytovat i dodatečné funkcionality, které jsou spojeny se získáváním dat, jejich propagací na klienta, zabezpečení, organizací komponent, lokalizací a skinováním komponent.
\subsection{Funkční požadavky}
Z dosavadního popisu problému byly vytvořeny následující požadavky na systém.
\begin{itemize}
\item Framework bude umožňovat generovat metadata objektů, na základě kterých budou generovány komponenty.
\item Framework bude umožňovat vygenerovat formulář nebo tabulku na základě dat získaných ze serveru.
\item Framework bude generovat metadata, která nebudou závislá na platformě.
\item Framework bude umožňovat získat data ze serveru.
\item Framework bude umožňovat naplnit formulář i tabulku daty.
\item Framework bude umožňovat odeslat data z formuláře zpět na server.
\item Framework bude umožňovat používat lokalizační resource bundly.
\item Framework bude umožňovat validaci dat na základě metadat, která obdržel od serveru.
\item Framework bude umožňovat klientovi překrýt chybové validační hlášky.
\item Framework bude umožňovat skinovatelnost.
\item Framework bude umožňovat specifikovat zdroje ve formátu XML.
\item Framework bude umožňovat vytvářet vstupní pole, combo boxy, výstupní pole, textarea, checkboxy, option buttony.
\item Framework bude umožňovat vkládat do formulářových polí texty, čísla a datum.
\item Framework bude umožňovat generování komponent určených pouze pro čtení. 
\end{itemize} 
Z požadavků vyplívá, že framework bude umožňovat získání definici dat na serveru a poté je distribuuje koncovému uživateli. Koncový uživatel tedy nebude potřebovat znát objekty, s kterými pracuje. Toto zaručí pružnou reakci na změnu dat a generování aktuálních formulářů či tabulek. 
\section{Popis architektury a komunikace}
Jak již bylo uvedeno, definice objektů, na základě které se budou vytvářet komponenty, je generována na serverové straně. Klient tedy komunikuje se serverem a vyžádá si tyto definice. Dále je potřeba definice na klientovi zpracovat. Definice nebudou závislé na platformě. Budou tedy popisovat data v obecné formě, což umožní generovat formuláře a tabulky nezávisle na platformě, jazyku a technologii. Referenční implementace bude napsána v jazyce Java s využitím komponentového frameworku Swing \cite{swing}. Definice dat, která jsou zasílána ze serveru na klienta by neměla ovlivňovat ostatní klienty, kteří framework nepoužívají. 

Business proces, který zachycuje generování formuláře včetně validace a odeslání je zachycen na obrázku \ref{img:businessModel}. Uživatel nejprve specifikuje zdroje, které bude klient využívat. Rozeznáváme následující zdroje:
\begin{enumerate}
\item Zdroj s metadaty, které definují komponentu.
\item Zdroj s daty, která budou v komponentech zobrazena.
\item Zdroj, na který budou data odeslána.
\end{enumerate}
Následně je vygenerována komponenta na základě metadat. V případě, že byl specifikován zdroj s daty, klientská část aplikace požádá server o tato data a vloží je do předpřipravené komponenty. Pokud datový zdroj specifikován není, zůstane komponenta bez konkrétního obsahu. Komponenta je nyní připravena a uživatel s ní může pracovat. V případě, že uživatel chce odeslat data na server a specifikoval zdroj, na který budou data odeslána, pak framework provede validaci dat. Pokud je validace dat úspěšná, na základě metadat se sestaví objekt, který je naplněn daty z aktuálního formuláře, a je odeslán na server. Server zpracuje request a vrátí klientovi odpověď. Na základě této odpovědi může uživatel dále upravovat formulář či s ním pracovat.

\subsection{Metadata}
Metadata \cite{metadata} jsou data o datech. Framework, který je popsán v této práci generuje na serverové straně metadata a zasílá je na klienta, který na jejich základě sestaví komponenty a poté je s nimi schopný pracovat. Klient musí být schopný zpětně sestavit data a odeslat je na server. Klient využívá klientskou část frameworku a server serverovou část. Z hlediska implementace je důležité, aby byly objekty, nesoucí informace o metadatech, stejné a bylo možné provést na klientovi generování na základě těchto dat.
Následující doménový model, modelovaný pomocí UML \cite{UmlArlow}, znázorňuje popis definic objektu, který je vytvořen po inspekci zadaného objektu. Inspekce vytvoří XML popis, který je převeden na obecný popis, jenž lze využít ke generování dat na klientovi. Tento obecný popis je zaslán klientovi, který využívá klientskou část frameworku, jenž očekává tyto objekty a na jejich základě je schopná vygenerovat uživatelské rozhraní. Na obrázku \ref{img:metadataModel} je zobrazen doménový model \cite{UmlArlow}, který je použit při popisu metadat objektu. Nejedná se o doménový model frameworku, ale pouze jeho části, která je zodpovědná za reprezentaci metadat. 

\begin{figure}[h!]
\includegraphics{images/domainModel}
\caption{Doménový model objektů obsahující metadata o objektu, nad kterým byla provedena inspekce}
\label{img:metadataModel}
\end{figure}

\subsubsection{AFMetaModelPack}
Tato třída zapouzdřuje informace, které popisují objekt, nad kterým byla prováděna inspekce. Třída rovněž slouží jako fasáda a nabízí programátorovi upravení metadat po generování. V případě serveru je toto návratový typ zdroje, na který klient přistoupí, chce-li znát metadata, která zdroj poskytuje.
\subsubsection{AFClassInfo}
Tato třída udržuje informace o hlavním objektu, z kterého je vytvářena definice. Třída má dále reference na své vnitřní proměnné a své potomky. Na základě generování dat je potřeba udržet pořadí, v kterém byla nad jednotlivými komponenty prováděna inspekce. V případě inspekce dat je i reference na neprimitivní datový typ jeho proměnná. Nicméně je potřeba, aby byl klient schopný určit, že se jedná o složitý datový typ a určit jeho pořadí v aktuálním objektu, aby mohl rozhodnout na jaké pozici objekt zobrazit. Z tohoto důvodu je ve třídě AFFieldInfo proměnná classType, která určuje, zdali se jedná o vnitřní třídu či primitivní datový typ.
\subsubsection{AFFieldInfo}
Tato třída je zodpovědná za poskytování detailních informací o proměnné objektu, nad kterým byla provedena inspekce. Třída udržuje název proměnné, widget, na který bude proměnná převedena, pravidla, která musí být splněna, název, pod kterým bude prezentována uživateli a zdali se jedná o složitý či jednoduchý objekt. Kromě těchto vlastností nese objekt také informace o tom, je-li komponenta viditelná a pouze pro čtení.
\subsubsection{AFRule}
Každá proměnná má souhrn vlastností, které musí být splněny. Typ widgetu ještě vždy nemusí určovat datový typ komponenty a neurčuje, je-li pole povinné či nikoliv. Tento soubor vlastností je popisován v této třídě. Třída využívá ENUM, který specifikuje podporované validace. Důvodem je to, že klient musí být schopný vytvořit tyto validační pravidla a interpretovat je na komponentě svým vlastním způsobem, který je specifický pro technologii, kterou používá. Jednou z dalších výhod je validace XML souboru, z kterého jsou pravidla vytvářena. Framework vytvoří pouze ty validační pravidla, která podporuje. Klient poté musí podporovaná pravidla interpretovat. 
\subsubsection{AFOptions}
Některé widgety umožňují, aby si uživatel vybral z několika předem připravených možností. Tyto možnosti musí být klientovi prezentovány. Tato třída udržuje informace o možnostech výběru v dané komponentě. Proměnná klíč je hodnota, která bude odeslána zpět na server a proměnná value je hodnota, která bude zobrazena klientovi. Tímto způsobem lze klientovi zobrazit jakýkoliv text, který bude zpětně mapován na jeho skutečnou hodnotu. Kromě textu lze samozřejmě zobrazit čísla či hodnoty výčtových typů.

\subsection{Server}
Server je zařízení či software, který umožňuje zpracovat požadavky od klientů a na jejich základě vytvořit odpověď. Server tedy poskytuje svým klientům určitý typ obsahu. Způsob a původ obsahu, který server poskytuje, je pro klienta povětšinou neznámý. V současné době je velmi populární přístup, při kterém server získá data z více zdrojů a poskytne je klientovi. Hovoříme o tzv. mashup \cite{Tuchinda2008}. Mashup nemusí být pouze z veřejných zdrojů, lze využít i privátní zdroje, či lze k sestavení odpovědi využít další služby. Klient napojený na server tohoto typu nemusí mít o těchto dalších zdrojích vůbec žádnou povědomost a dotazuje se pouze vůči tomuto serveru, který zpracovává jeho požadavky. 
Klientů, kteří získávají data z veřejných, či privátních zdrojů serveru může být celá řada. Mohou to být vlastní privátní aplikace, mobilní aplikace, Javascriptoví klienti či další server, který pouze využívá veřejné zdroje serveru k sestavení odpovědi svým vlastním klientům. V těchto případech je potřeba zvážit způsob generování definic dat, které by mohly způsobit stávajícím klientům problémy. V ideálním případě musí být framework integrován takovým způsobem, aby byla zachována stávající funkcionalita a framework ji pouze rozšířil. 
Ke generování definic objektů jsou potřeba následující věci:
\begin{enumerate}
\item Objekt, jehož definice budou generovány.
\item Mapování, na jehož základě bude rozhodnuto, o jaký typ komponenty půjde.
\item Definice komponenty včetně vlastností jako jsou validace, layout a popis chování komponenty.
\item Layout, ve kterém budou komponenty sestaveny.
\item Framework, který provede inspekci.
\item Framework, který bude inspekci řídit a bude interpretovat vygenerovaná data. Tento framework musí zároveň ověřit validitu jednotlivých komponent.
\end{enumerate}

Výše uvedené vlastnosti, nebudou mít vliv na změnu funkcionality. K inspekci a mapování bude využit framework AspectFaces \cite{aspectFaces}, který umožňuje na základě datových typů rozhodnout jakou komponentu využít. Definice komponent a jejich vlastností bude již v plné kompetenci vývojáře, nicméně základní komponenty a jejich chování bude předpřipraveno ve vzorovém projektu, aby se vývojář mohl inspirovat.

\subsection{Klient}
Klientská část frameworku bude vytvářet komponenty na základě metadat, která obdržela od serveru. Klient nebude mít žádnou znalost o objektech, které mu server poskytuje, předtím než obdrží jejich definice. Klient nicméně musí vědět, který zdroj mu poskytne relevantní definice a který ze zdrojů mu poskytne data odpovídající těmto definicím. Zároveň také musí vědět, na který zdroj data zpětně odeslat. Zdroj je obvykle specifikován následujícími parametry:
\begin{enumerate}
\item Adresa serveru
\item Port
\item Protokol
\item Metoda (get, post, put, delete)
\item Dodatečné hlavičkové parametry například content-type
\end{enumerate}
Klient tedy bude muset vždy specifikovat tyto parametry. Z hlediska použitelnosti je vhodné mít tyto specifikace v XML souboru, který bude umět klient jednoduše načíst. Pro usnadnění bude načítání provádět framework. Ukázka je na obrázku \ref{code:xmlSource}. V ukázkovém příkladu je specifikován zdroj s metadaty, který je vždy povinný. Zdroj se nachází na adrese\\ http://localhost:8080/AFServer/rest/users/loginForm. Zdroj s daty není specifikován, což způsobí, že ve formuláři nebudou žádná data. Formulář bude možné odeslat na adresu http://localhost:8080/AFServer/rest/users/login. Zdroj má identifikátor loginForm. V jednom XML dokumentu lze mít více zdrojů. K vložení dat do konkrétního zdroje lze využít EL. V hlavičce může být 0 až N parametrů, přičemž každý parametr musí být uveden ve stejném formátu, jako je znázorněno na obrázku \ref{code:xmlSource}. Obdobný způsob se využívá v JavaEE aplikací v deskriptoru web.xml. Klient umí sestavit požadavek na základě tohoto popisu a interpretovat odpověď od serveru. Není tedy nutné, aby uživatel implementoval třídy, které se umožní aplikaci připojení na server a získání dat.

\begin{lstlisting}[caption=Ukázka XML specifikace zdrojů,
label={code:xmlSource}, basicstyle=\footnotesize]
<?xml version="1.0" encoding="UTF-8"?>
<connectionRoot xmlns:xsi="http://www.w3.org/2001/XMLSchema-instance">
	<connection id="loginForm">
		<metaModel>
			<endPoint>localhost</endPoint>
			<endPointParameters>/AFServer/rest/users/loginForm</endPointParameters>
			<protocol>http</protocol>
			<port>8080</port>
			<header-param>
				<param>content-type</param>
				<value>Application/Json</value>
			</header-param>
		</metaModel>
		<send>
			<endPoint>localhost</endPoint>
			<endPointParameters>/AFServer/rest/users/login</endPointParameters>
			<protocol>http</protocol>
			<port>8080</port>
			<method>post</method>
			<header-param>
				<param>content-type</param>
				<value>Application/Json</value>
			</header-param>
		</send>
	</connection>
</connectionRoot>
\end{lstlisting}

Je patrné, že klient je schopný získat definice formulářů či tabulek, naplnit je daty a poté odeslat zpět na server. Důvodem, proč je klient schopný generovat formuláře na základě definice ze serveru je ten, že klient pracuje se stejnými objekty, které popisují metadata, jako server. Prozatím je k dispozici pouze strohý popis dat. Klientská část musí nyní rozhodnout jak data interpretovat, jak s nimi pracovat, jakým způsobem je validovat, jak je znovu sestavit a odeslat na server. Důležitým prvkem je i způsob uspořádání jednotlivých prvků, jejich velikosti, barvy a texty.

Využití frameworku by obdobně jako v případě serveru nemělo mít vliv na stávající použití aplikace. V případě referenčního řešení ve Swingu generuje klient JPanel, který lze vkládat do dalších panelů a vývojář tak není nikterak omezen, co se týče stávající aplikace. 

\begin{figure}[h!]
\includegraphics{images/formLifecCycle}
\caption{Životní cyklus formuláře}
\label{img:formLifeCycle}
\end{figure}

\subsection{Životní cyklus formuláře}
Formulář, jako každá komponenta, má svůj životní cyklus. Jeho stavy jsou znázorněny na obrázku \ref{img:formLifeCycle}. Formulář je po vygenerování a po naplnění daty v inicializačním stavu. V tomto stavu může být komponenta nevalidní, neboť data, která obdržela, nemusí splňovat požadované validace. K této situaci může dojít, je-li například přidána nová proměnná do datového modelu. Toto přidání obvykle probíhá tak, že se v koncové databázi, pokud jí software má, přidá políčko a nastaví se mu defaultní hodnota pro již existující data, která může být například null. V definici může být pole označeno jako povinné, avšak ve formuláři nemusí být vyplněno, což způsobí, že jsou data nevalidní, byť uživatel ve formuláři data nezměnil. Komponenta se pak přepne do stavu Inconsistent. V případě modifikace dat se komponenta dostává do stavu Detached. Tento stav značí, že byla data změněná. Pokud uživatel data stále mění, pak komponenta zůstává v tomto stavu. Ze stavu Detached přechází komponenta v případě úspěšné validace do stavu Consistent. V případě neúspěšné validace se komponenta dostane do stavu Inconsistent. Z tohoto stavu lze přejít pouze do dvou stavů jedním z nich je stav Detached. Do tohoto stavu lze přejít, pokud uživatel změní data ve formuláři. Komponenta může také zůstat ve stavu Inconsistent, pokud uživatel data nezmění a zkusí provést validaci znovu a tato validace opět selže. Konzistentní stav značí, že je komponenta připravená k odeslání dat na server. Pokud odeslání dat selže je komponenta přepnuta do stavu Inconsistent. V případě úspěšného odeslání dat, je komponenta přepnuta do stavu Synchronized. Ze stavu Synchronized lze přejít do stavu Initialized, pokud jsou data znovu načtena ze serveru a do stavu Detached, pokud jsou data upravena uživatelem.

\section{Případy užití}
Případy užití a jejich scénáře \cite{UmlArlow} specifikují chování systému. V této práci lze nahlížet na případy užití ze dvou stran. První z nich je koncový uživatel, neboli vývojář, který framework využívá. Druhým z nich je samotný framework, který provádí akce, aby splnil úkol, který mu uživatel uložil. V této sekci se zaměříme na případy užití koncového uživatele, které specifikují použití frameworku. Na obrázku \ref{img:useCase} jsou zachyceny všechny tyto případy užití. Pro ukázku detailně rozebereme případ užítí na obrázku \ref{img:useCaseSmall}. Na tomto případu je znázorněno odeslání dat z vygenerovaného formuláře zpět na server. Součástí je samozřejmě validace zadaných dat a jejich zpětné sestavení, neboť formulář byl vytvořen na základě metadat a klient tedy zná pouze strukturu objektu popsanou těmito metadaty. 
\begin{figure}[h!]
\begin{center}
\includegraphics{images/useCaseSmall}
\caption{Část případů užití znázorňující odeslání dat na server z vygenerovaného formuláře}
\label{img:useCaseSmall}
\end{center}
\end{figure}
\subsubsection{Validace komponenty}
Tento případ užití je znázorněn na obrázku \ref{img:useCaseSmall} a jmenuje se ValidateComponent.

Případ užití: Validace komponenty\\
ID: 1\\
Popis: 
Uživatel využívá framework ke generování formulářů. Hlavním úkolem formuláře je možnost vkládat či upravovat data a odesílat je zpět na server. Před odesláním dat na server musí být provedena validace, aby se zajistilo, že bude formát dat serveru vyhovovat a bude umět s daty pracovat.
\\
Aktér: Uživatel\\
Vstupní podmínky:
\begin{enumerate}
\item Formulář musí být sestaven na základě metadat a framework musí znát formulář, s kterým chce pracovat.
\end{enumerate}
Scénář:
\begin{enumerate}
\item Případ užití začíná obdržením požadavku od uživatele žádající validaci dat.
\item Systém vyhledá pole k validaci.
\item Dokud existují pole k validaci pak:
\begin{enumerate}
\item Systém získá konkrétní pole k validaci.
\item Systém určí typ komponenty a požádá builder, který ji sestavil o data.
\item Dokud existuje validace, která zatím nebyla na poli vykonána pak:
\begin{enumerate}
\item Systém požádá validátor o validaci.
\item Pokud validace selže pak:
\begin{enumerate}
\item Systém ukončí validování tohoto pole a zobrazí u něj chybové validační hlášení.
\end{enumerate}
\end{enumerate}
\end{enumerate}
\end{enumerate}

Případ užití: Sestavení dat\\
ID: 2\\
Popis: 
Uživatel využívá framework ke generování formulářů. Hlavním úkolem formuláře je možnost vkládat či upravovat data a odesílat je zpět na server. Před odesláním dat na server musí být tyto data zpětně sestaveny, aby s nimi server dokázal pracovat.
\\
Aktér: Uživatel\\
Vstupní podmínky:
\begin{enumerate}
\item Formulář musí být sestaven na základě metadat a framework musí znát formulář, se kterým chce pracovat. Framework musí znát formát dat, která server očekává.
\end{enumerate}
Scénář:
\begin{enumerate}
\item Případ užití začíná obdržením požadavku od uživatele žádající sestavení dat z formuláře.
\item Zahrnout(Validace komponenty).
\item Pokud validace dopadla úspěšně pak:
\begin{enumerate}
\item Systém získá pole, která budou odeslána.
\item Pokud existuje pole, které ještě nebylo transformováno pak:
\begin{enumerate}
\item Systém určí typ komponenty a požádá builder, který ji sestavil, o data.
\item Systém určí název proměnné a třídu, do které patří, a nastaví jí data.
\end {enumerate}
\item Systém na základě formátu dat, které server očekává, rozhodne, v jakém formátu data zaslat a převede je na daný formát.
\end{enumerate}
\end{enumerate}

Výstupní podmínka:
\begin{enumerate}
\item Data ve formuláři byla převedena na objekt, s kterým umí server pracovat.
\end{enumerate}

Případ užití: Odeslání dat\\
ID: 3\\
Popis: 
Uživatel využívá framework ke generování formulářů. Hlavním úkolem formuláře je možnost vkládat či upravovat data a odesílat je zpět na server. Před odesláním dat na server musí být tato data zpětně sestavena, aby s nimi server dokázal pracovat, a musí splnit validační kritéria.
\\
Aktér: Uživatel\\
Vstupní podmínky:
\begin{enumerate}
\item Formulář musí být sestaven na základě metadat a framework musí znát formulář, se kterým chce pracovat. Framework musí znát zdroj, na který mají být data odeslána, a všechny potřebné informace, které zdroj vyžaduje.
\end{enumerate}
Scénář:
\begin{enumerate}
\item Případ užití začíná obdržením požadavku od uživatele žádající odeslání formuláře na server.
\item Zahrnout(Validace komponenty).
\item Zahrnout(Sestavení dat)
\item Dokud je validace či sestavení dat neúspěšné pak:
\begin{enumerate}
\item Systém zobrazí chybové hlášení a určí, u kterých polí nastala chyba.
\item Uživatel chybu opraví a požádá systém o opětovnou validaci a sestavení dat.
\end{enumerate}
\item Systém vytvoří připojení na specifikovaný zdroj a odešle data.
\item Pokud odeslání selhalo pak:
\begin{enumerate}
\item Systém zobrazí chybové hlášení, že nebylo možné data odeslat včetně odpovědi od serveru, je-li nějaká.
\end{enumerate}
\end{enumerate}

Výstupní podmínka:
\begin{enumerate}
\item Data byla odeslána.
\end{enumerate}


\section{Omezení frameworku}
Existují určité možnosti, které nebudou ve frameworku podporovány. V následujícím přehledu budou představeny nepodporované vlastnosti frameworku v aktuální verzi.
\begin{enumerate}
\item Inspekce datových proměnných typu List a Array
\item Získávání dat ve formátu XML. Framework plně podporuje JSON.
\item Customizace jednotlivých polí. Framework podporuje customizaci formuláře a všech jeho polí, nicméně nedisponuje možností přizpůsobovat jednotlivá pole.
\item Framework neumožňuje v jednom poli reprezentovat složený datový typ. Například třídu.
\item Odesílání dat a jejich validace z tabulky. Tabulka je v této verzi pouze readonly.
\item Framework vyžaduje ke své funkčnosti jak serverovou tak klientskou stranu. V případě, že klientské straně chybí serverová strana, tak je framework nefunkční. V případě, že serverové straně chybí klientská strana, pak je serverová strana stále schopná vytvářet definice dat.
\item Klientská strana zobrazuje pouze ta data, která obdržela od serveru, nelze vytvářet automatický Mashup na klientovi. Nicméně klient může generovaný formulář umístit mezi jiné komponenty.
\item Klientská strana nedokáže sestavit objekt, který získala z metadat do takové míry aby z něj byla schopná vytvořit instanci. Neboli klientská strana si neudržuje konkrétní objekt, který obdržela, ale pouze jeho popis.
\item Serverová část využívá k automatické inspekci framework. Bez tohoto frameworku není možné inspekci provést, nicméně uživatel si může definovat svou vlastní definici bez nutnosti inspekce dat.
\end{enumerate}

\section{Uživatelé a zabezpečení}
Téměř každá aplikace využívá způsob, při kterém se uživatel autentizuje a aplikace mu na základě jeho rolí přidělí oprávnění. V případě využití bez stavového protokolu, jakým REST je, lze posílat informace o uživateli v hlavičce požadavku. Tyto informace mohou být samozřejmě zašifrované. Framework podporuje vkládání libovolných informací do hlavičky requestu. Klient si také může zvolit, zdali bude využívat http či https protokol. Velmi rozšířenou možností je využitý Oauth. Jednou z možností je vložení parametrů do hlavičky, či do adresy. Vkládání dynamických adres či proměnných do hlavičky requestu framework podporuje. 

Druhou částí jsou uživatelské role, na základě kterých se generuje uživatelské rozhraní. Serverová část využívá framework AspectFaces \cite{aspectFaces}, který podporuje uživatelské role v systému. Je jen na programátorovi, jaký framework na autentizaci a autorizaci na straně serveru použije. Jednou z možností je například napsat si vlastni interceptor, který určí, o jakého uživatele se jedná, přiřadí mu roli v systému, na základě které se mu zobrazí konkrétní obsah. Server při generování metadat může využít různé mapovací soubory na základě uživatelské role. Specifikace tohoto chování je opět v plné kompetenci uživatele.
\section{Použité technologie}
V následující sekci jsou rozebrány použité technologie. Kromě samotného frameworku je součástí práce i ukázkový projekt na platformě JavaEE a JavaSE.
\subsection{Java SE - Swing}
Klientská část frameworku je schopná vygenerovat formuláře nebo tabulky, naplnit je daty a data odeslat. Klientská část je přizpůsobena frameworku Swing. Důvodem je, že vývoj Swingové aplikace je rychlý a Swing zná velké množství vývojářů, kteří si framework mohou jednoduše otestovat. Nicméně metadata, která server generuje lze interpretovat v jakékoliv technologii. Swing je knihovna grafických a uživatelských prvků. Poskytuje komponenty, layouty, actionListenery, okna, dialogová okna a další prvky, pomocí kterých lze vytvářet interaktivní aplikace.
\subsection{Java EE}
Java EE je platforma sloužící k vývoji enterprise aplikací. V současné době je oblíbená jak u velkých tak u menších korporací. Java EE přináší podporu pro Restové služby, JFS, JSP, EJB, databázové frameworky, anotace a další komponenty. Aplikace v JavaEE se obvykle nasazuje na aplikační server. Aplikační servery mohou být v cloudu a podílet se společně na zpracování requestů. Důvodem využití této platformy je fakt, že klient vytváří requesty vůči serveru a server tato data zpracovává. Je vhodné mít na serveru platformu, která je ověřená a má potenciál ke zpracování těchto requestů. V této práci generujeme data pomocí restového rozhraní a Java EE splňuje specifikaci, které se tohoto rozhraní týká. 
\subsection{AspectFaces}
AspectFaces je framework, který umí provádět inspekci nad zadanými objekty a na základě datových typů a dalších parametrů rozhodovat o tom, jaká komponenta se použije pro konkrétní datový typ. Framework využívá AspectFaces k tomu, aby provedl toto mapování a sestavil popis uživatelského rozhraní. Hlavním důvodem využití tohoto frameworku je fakt, že je distribuován jako open source pod licencí LGPL v3 a lze využít jeho funkcionalitu ke statické inspekci dat. Tato inspekce je již odladěna a není tedy důvod psát znovu již vynalezenou věc. 
\subsection{Ukázkový projekt}
Ukázkový projekt demonstruje použití frameworku. Skládá se ze dvou částí. Klientské a serverové. Klientská část využívá pouze Swing. Serverová část je mnohem sofistikovanější a využívá aktuální technologie. Ukázkový projekt zde znázorňuje použítí frameworku a jeho omezení. Ukázkový projekt je koncipován tak, aby ho bylo možné nasadit bez nutnosti dodatečného nastavení.
\subsubsection{GlassFish}
GlassFish \cite{glassfish} je open source aplikační server. Jedná se o certifikovaný server JavaEE, umožňující clustering, monitoring, podporu EJB, REST a JDBC. Architektura jádra je založena na frameworku OSGI, který umožňuje vzdáleně přidávat, startovat či ukončovat komponenty bez nutnosti restartování celého serveru. Důvodem, proč je Glassfish využit na tomto projektu je čistě demonstrativní. Nicméně každý aplikační server má svá specifika, a proto je ukázková aplikace odladěna právě pro GlassFish. Jak již bylo zmíněno, Glassfish distribuje vestavěnou podporou pro REST a umožňuje použití Derby DB v režimu in-memory bez nutnosti speciálního nastavení.
\subsubsection{RestEasy}
Jedná se o framework, pomocí kterého lze vytvářet RESTful aplikace. Tento framework může běžet v libovolném servletovém kontejneru. Framework podporuje například JSON, XML serializace objektů, EJB a je splňuje JAX-RS implementaci. Tvorba aplikací s restovým rozhranním je tak díky tomuto frameworku mnohem jednodušší a vývoj je rychlejší. 
\subsubsection{EJB}
Enterprise JavaBeans \cite{javaEETutorial} jsou serverově orientované komponenty, které zapouzdřují business logiku a přístup do databáze. Jsou spravovány v rámci serverového kontejneru, který zajišťuje jejich vytvoření i odstranění z paměti. EJB mohou být různých typů.
\begin{itemize}
\item Stateless
\item Statefull
\item Singleton
\end{itemize}
Jak již bylo zmíněno, o jejich správu se stará serverový kontejner, nemusíme tedy řešit problémy spojené s vytvořením a destrukcí singletonu \cite{gamma}. Mezi hlavní výhody EJB patří transakční zpracování, zajištění systémových služeb a bezpečnostní autorizace. Abychom definovali či~získali přístup k těmto třídám, používáme anotace, které jsou velmi dobře čitelné a~srozumitelné.
\subsection{Derby DB}
V ukázkovém projektu je potřeba data ukládat do databáze. Při vytváření byl kladen důraz na to, aby noví uživatelé nemuseli v konfiguračních souborech specifikovat nastavení, tudíž by mohli ukázkový projekt ihned nasadit a vyzkoušet. Z tohoto důvodu je využita light databáze Derby. Ukázkový projekt ji využívá v in-memory módu, což znamená, že budou data po zastavení serveru ztracena. Spolu s Derby DB využívá ukázkový projekt ORM s defaultním nastavením na create. Po startu aplikace jsou v čisté databázi vytvořeny požadované tabulky, které odráží definice objektů, jenž jsou anotované jako entity. Další výhodou je možnost využít anotací k nastavení validací přímo na databázi. Tyto validace pak mohou být využity při inspekci dat a na jejich základě mohou být vytvořeny validace, či konkrétní komponenty.
\chapter{Implementace} 
\section{Architektura}
V tomto frameworku rozlišujeme klientskou a serverou část. Serverová část generuje data pro klienta a tímto způsobem ovlivňuje ovládací prvky, které klientská část aplikace zobrazuje uživateli. Diagram nasazení na obrázku \ref{img:deploymentFrameworkDiagram} zachycuje použití frameworku. Serverová část frameworku je nasazena na klientovi a je schopná generovat definice formulářů s použitím frameworku AspectFaces \cite{aspectFaces}. Tyto definice jsou převedeny na model, který je možné upravit a odeslat klientovi. Aby byla serverová část plně funkční je potřeba nasadit aplikaci, v které je využívána na Java EE aplikační server. Nicméně v případě využití pouze staticky generovaných definic, lze aplikaci nasadit na libovolný aplikační server, který bude poskytovat klientům definice kompatibilní s definici poskytovanými při dynamickém generování. Specifikace formuláře, je poté zaslána na klienta, který ji interpretuje za použítí klientské částy zvané AFSwinx. Tato část využívá i serverovou část a to z důvodu kompatibilnosti objektů a jejich vlastností. Přidání do projektu lze provést tak, že se do adresáři s knihovnami vloží přeložený jar soubor, či se přidá projekt jako Maven závislost. V současné době není framework k dispozici v centrálním repositáři, je tedy potřeba stáhnout aktuální verzi a zkompilovat ji do lokálního repositáře.  
\subsection{Server}
Jak již bylo zmíňěno, tak server využívá ke generování serverou část frameworku nazvanou AFRest. Na obrázku \ref{img:serverSide} jsou zobrazeny třídy a balíčky, které tato část využívá. Jsou zde výčtové typy, které určují podporované komponenty a jejich vlastnosti, dále objekty zodpovědné za informace o volbě layoutu a objekty nesoucí informace o definicích, na základě kterých budou sestaveny formuláře či tabulky klientem a samozřejmě třídy zodpovědné za inspekci dat. Framework doplňuje do AspectFaces několik anotací, které lze využít při generování definic. Jsou to následující anotace:
\begin{enumerate}
\item @UIWidgeType - tato anotace určuje typ widgetu, který se použije do xml šablon, které se používají při generování definic je propagován jako proměnná s názvem widgetType
\item @UILayout - tato anotace definuje layout na dané proměnné. Lze specifikovat typ layoutu, jeho orientace a pozice popisu prvku. Do xml šablon jsou tyto hodnoty propagovány jako layout, layoutOrientation a labelPossition. 
\end{enumerate}
Výše zmíněné anotace akceptují pouze hodnoty z výčtových typů v balíčku common. V případě typu komponenty nebo-li widgetType přijímá anotace hodnoty ze třídy SupportedWidgets a v případě anotace určující layout lze vložit pouze hodnoty z výčtových typů LayoutDefinitions, LayoutOrientation a LabelPosition. Hlavní výhodou tohoto řešení, je typová kontrola a jistota, že klient obdrží od serveru pouze takové hodnoty, s kterými je schopný pracovat. Stejným principem jsou řešeny validace a proměnné, které definují vlastnosti jednotlových komponent. 

\subsubsection{Generování modelu}
Výsledkem inspekce objektu je model, který nese informace potřebné k tomu, aby klient mohl sestavit formulář či tabulku a byl do těchto komponent schopný vložit data získaná ze serveru. Na obrázku \ref{img:metaModelFinal} je konečná podoba modelu, který je vytvořen k tomuto účelu. Model je výsledkem hledání analytických tříd z doménového modelu na obrázku \ref{img:metadataMode}. Tento model již byl posán v analytické části, nicméně v této části je již model kompletní a proto zde budou uvedeny pouze změny oproti původnímu modelu. Proměnné, které nesou informace o layoutu, typu komponenty validátorech a jejich typech jsou výčtové typy. Jak již bylo zmíněno výhodou je typová bezpečnost a jednoznačnost vlastností, které framework podporuje. Model slouží také jako fásada, k nastavení dodatečných atributů. Jedním z těchto atributů je proměnná options ve třídě AFFieldInfo. Tato proměnná drží informace o možných hodnotách, které může komponenta nabývat. V současné verzi je tento atribut využit u komponent výběrového typu, mezi které patří například zaškrtávací políčka, či výběrová menu. Programátor specifikuje množinu těchto hodnot, v které klíč určuje hodnotu, jenž bude odeslána na server a text, který bude zobrazen uživateli je určen proměnnou value. Tyto možnosti nejsou generovány automaticky a v případě potřeby je musí programátor specifikovat ručně a to tak, že určí množinu dat a pole, ke kterému je přiřazeno. Třída AFMetaModelPack poskytuje zapouzdřuje způsob jakým se množina dat nastaví na konkrétní políčko a nabízí uživateli funkci, která je schopná nastavení provést na základě dat, zadaných uživatelem.

K dynamickému generování definic se využívá framework AspectFaces \cite{aspectFaces}, který umožňuje na základě mapování rozhodnout jaká komponenta bude použita pro konkrétní proměnnou dané třídy. Dále nabízí určení layoutu, který bude použit a samozřejmě určení mapovacího souboru. Tímto lze docílit mnoha různých transformací. Tento framework je potřeba nejprve nastavit, nicméně toto nastavení provede za vývojáře serverová část frameworku AFRest. Rozhranní AFRest z obrázku \ref{img:metaModelFinal} a jeho implementace AFRestGenerator provedou kompletní nastavení a spustí generování dat. Rozhranní umožňuje uživateli určit mapovací soubor a template, která bude použita. Mapování lze použít na všechny promměné obejktu, či může vývojář určit, které mapování se použije na konkrétní proměnnou. Ukázka mapování z frameworku AspectFaces je na znázorněna v ukázce zdrojových kódu \ref{code:xmlMapping}. Proměnná typu String se bude mapovat na vstupní textové pole, kterýé je definováno v structure/inputField.xml, v případě že se bude jednat o typ password, tak se bude proměnná typu String mapovat na vstupní textové pole typu, které místo vepsaných znaků zobrazuje zástupné znaky, komponenta je definována v structure/inputPassword.xml. Typ Address, což je neprimitivní datový typ se bude mapovat na entitní typ, jehož definice je v structure/entity.xml.
\begin{lstlisting}[caption=Ukázka mapování proměnných na komponenty,
  label={code:xmlMapping}]
<mapping>
	<type>String</type>
	<default tag="structure/inputField.xml" maxLength="255"/>
	<condition expression="${type == 'password'}" tag="structure/inputPassword.xml" />
</mapping>
<mapping>
	<type>Address</type>
	<default tag="structure/entity.xml" />
</mapping>
\end{lstlisting}
Mapování tedy určí soubor s komponentou, který bude reprezentovat aktuální proměnnou. Soubor s definicí komponenty, je pak dále využit k finálnímu definici proměnné. Ukázka vstupního textového pole je v ukázce zdrojových kódu \ref{code:xmlInputField}. Komponenta je v kořenovém elementu widget. Jelikož se jedná pouze o fragment xml, který je použit ke složení celé definice, jenž je uvedena v příloze v ukázce zdrojových kódů \ref{code:xmlCompleteDefinition}, tak zde není uvedena deklarace XML \cite{xml}. Ve výsledném XML již však deklarace již uvedena je. Popis jednotlivých uzlů je v tabulce \ref{table:xmlComponentAttributes}.
\begin{lstlisting}[caption=Ukázka definice komponenty,
  label={code:xmlInputField}]
<widget>
	<widgetType>textField</widgetType>
	<fieldName>$field$</fieldName>
	<label>$label$</label>
	<validations>
		<required>$required$</required>
		<minLength>$minLength$</minLength>
		<maxLength>$maxLength$</maxLength>
	</validations>
	<fieldLayout>
		<layoutOrientation>$layoutOrientation$</layoutOrientation>
		<labelPossition>$labelPossition$</labelPossition>
		<layout>$layout$</layout>
	</fieldLayout>
</widget>
\end{lstlisting}
Knihovna AspectFaces umožňuje určovat způsob jakým bude prováděna inspekce. Tento způsob se určuje v šablonách. K optimálnímu využití je nejvýhodnější použít způsob, při kterém je provedena inspekce všech proměnných, které mají definováno mapování. V případě jednoduchých datových typů je vše v pořádku, nicméně knihovna neobsahovala nativní podporu pro neprimitivní datové typu, v případě že byla použita inspekce, která by nevyužívala JSF. Z tohoto důvodu je důležité, aby se všechny neprimitivní datové typy mapovali na entity.xml, která je znázorněna v části zdrojového kódu \ref{code:xmlEntity}. Framework totiž pro všechny tyto entity provede inspekci znovu a následně části sestaví sestaví a vznikne tak kompletní definice. V tomto bodě, lze určit mapování a šablony, které se mají při rekurzivní inspekci použít. Framework AspectFaces byl proto doplněn o proměnné, které umí vrátit kanoický název třídy a na základě tohoto názvu lze provést nad touto třídou inspekci. Jak je patrné z výsledné definice, tak každý uzel má svého rodiče. Na základě rodiče lze určit jednoznačně určit kam uzel patří. Tato vlastnost umožňuje provádět inspekci i nad třídami, které mají více proměnných stejného datového typu. Klient totiž potřebuje znát strukturu objektu, aby ho mohl zpětně sestavit a odeslat zpět na server, který objekt přijme. Znalost struktury klient taktéž vyžaduje v případě získávání dat.
\begin{lstlisting}[caption=Ukázka definice neprimitvního datového typu,
  label={code:xmlEntity}]
<entityClass>
	<entityFieldType>$DataTypeFullClassName$</entityFieldType>
	<fieldName>$fieldName$</fieldName>
</entityClass>
\end{lstlisting}
\subsubsection{Použití}
Aby byl klient schopný získat definice dat, tak musí serverová strana poskytovat zdroj těchto definic. V tomto zdroji server využije serverovou část frameworku ke generování dat. Použití je přímočaré a ukázka zdroje je zobrazena na v části zdrojového kódu \ref{code:serverDefinition}. Nejprve je vytvořena instance třídy AFrestGenerator, která umožňuje generování dat, jenž jsou následně odeslány klientovi. Generátor nastaví framework AspectFaces automaticky, nicméně očekává, že bude framework AspectFaces, použit. K správnému použití je potřeba, aby we WEB-INF byly konfigurační soubory a aby existovali mapovací soubory a definice komponent. V tomto případě využije implicitního nastavení pro mapování i šablony. Bude použito mapování v souboru structure.config.xml a šablona v template/structure.xml. Tyto ukázkové soubory jsou poskytovány spolu s frameworkem.

\begin{lstlisting}[caption={Ukázka zdroje, sloužícího k vygenerování definice třídy Country},
  label={code:serverDefinition}]
@GET
@Path("/definition")
@Produces({MediaType.APPLICATION_JSON})
@Consumes({MediaType.APPLICATION_JSON})
@RolesAllowed({"admin"})
public Response getResources(@javax.ws.rs.core.Context HttpServletRequest request) {
	try {
		AFRest afRest = new AFRestGenerator(request.getSession().getServletContext());
		AFMetaModelPack data = afRest.generateSkeleton(Country.class.getCanonicalName());
		return Response.status(Response.Status.OK).entity(data).build();
	} catch (MetamodelException e) {
		return Response.status(Response.Status.INTERNAL_SERVER_ERROR).build();
	}
}
\end{lstlisting}
Zdroj poskytuje definice dat. V případě, že klient požaduje data do vygenerované definice, tak je potřeba poskytnout klientovi objekt stejného typu, nad kterým byla prováděna definice, nebo objekt se stejnými proměnnými a datovými typy. V tomto případě třídu Country. Klientská strana nerozlišuje datový typ obdrženého objektu, avšak očekává, že objekt bude mít určité proměnné, ke kterým se budou vázat specifické validace. V některých případech je žádoucí, aby se na úrovni business a view nepracovalo s databázovou entitou, ale s jejím mapovacím objektem. V části zdrojových kódu \ref{code:serverData} je příklad získání dat do již vygenerovaného formuláře či tabulky. Zdroj využije EJB managera CountryManager k získání konkértní instace třídy Country z databáze. Tuto instanci vrátí klientovi. Tento zdroj nemá žádnou vazbu na předchozí zdroj, který generoval definice. Vývojář tedy v případě použítí frameworku nemusí měnit stávající implementaci, pokud již nějaká existuje. Stejně tak nemá použití frameworku dopad ne klienty, kteří již používají webové API serveru. Zodpovědnost za správnou interpretaci dat je na klientské straně.
\begin{lstlisting}[caption=Zdroj poskytující konrétní instanci třídy Country,
  label={code:serverData}]
@GET
@Path("/{id}")
@Produces({MediaType.APPLICATION_JSON})
@Consumes({MediaType.APPLICATION_JSON})
public Response getCountry(@PathParam("id") int id) {
	try {
		CountryManager<Country> countryManager = getCountryManager();
		Country country = countryManager.findById(id);
		return Response.status(Response.Status.OK).entity(country).build();
	} catch (BusinessException e) {
		return Response.status(Response.Status.BAD_REQUEST).build();
	} catch (NamingException e) {
		return Response.status(Response.Status.INTERNAL_SERVER_ERROR).build();
	}
}
\end{lstlisting}

\begin{table}[width=\linewidth]
\begin{center}
\caption{Uzly XML, které definují strukturu dat}
\label{table:xmlComponentAttributes}
\begin{tabular}{|p{7cm}|p{7cm}|}
\hline
\textbf{Uzel} & \textbf{Popis} \\
\hline
widget & 
Typ komponenty. Určuje jak komponentu bude klient interpretovat.\\
\hline
fieldName &
Název aktuální proměnné, kterou komponenta zastupuje.\\
\hline
Label &
Popis komponenty, který bude zobrazen uživateli. \\
\hline
validations &
Validace, které bude umět komponenta ověřit. \\
\hline
fieldLayout&
Popis layoutu, který bude na komponentě použit. \\
\hline
\end{tabular}
\end{center}
\end{table}


\subsection{Klient}
Klientská část aplikace, využívá klientskou část frameworku ke generování formulářů či tabulek. Definice a data získává ze server. Referenční implementace je napsána pro standalone aplikace na platformě Java SE s využitím technologie Swing. Integrace frameworku do klientské aplikace je možná dvěmi způsoby. Prvním z nich je vložení knihovny do složky lib a druhým je přidání Maven závislosti. 
\subsubsection{Komponenty}
Klientská část umožňuje generovat tabulky nebo formuláře. Tyto celky označujeme jako komponenty. V případě formuláře se tato komponenta skládá z dalších aktivních ovládacích prvků. Komponenty jsou oděděny z třídy AFSwinxTopLevelComponent, která implementuje rozhraní AFSwinxInteraction, jenž vynucuje implementovat metody k získání modelu, dat a k jejich odeslání zpět na server. Součáští je také validace dat. Mimo tohoto rozhraní používá implementuje třída ještě rozhranní ComponentResealization. Toto rozhranní je využito k zpětnému získání dat z komponent. Aby bylo možné přidávat komponenty do již existující aplikace, tak tato komponenta ještě dědí od třídy JPanel, což zajistí, že výslednou komponentu lze přidat na jakékoliv místo ve stávající Swingové aplikaci. Vývojář může nad takto generovanými komponentami provádět operace. V případě odeslání dat na server, lze tuto akci vyvolat metodou sendData. Komponenta již sama provede validaci dat, sestavení dat a jejich odeslání.

Při návrhu jsem se zaměřil i na použitelnost, neboť je potřeba aby framework umožňoval dodatečná nastavení, nicméně pokud se vývojář bude s frameworkem učit, tak je pravděpodobné, že bude chtít vytvořit první prototyp, aby si vyzkoušel funkčnost. Z tohoto důvodu byla zavedena třída AFSwinx, která slouží jako správce komponent. Umožňuje komponenty vytvářet, přidávat, mazat nastavovat globální skin a lokalizace. Důležitou součástí je i získání již sestavené komponenty. Každá komponenta je jednoznačně určena svým identifikátorem, který si vývojář zvolí. Na základě tohoto identifikátoru je zaregistrována a lze k ní získat přístup a provádět nad ní operace. Vzhled jednotlivých prvků v komponentě již není možné po vygenerování měnit. Skiny a lokalizace musí být tedy nastaveny před samotným vygenerováním. Také je potřeba určit způsoby připojení ke zdrojům a jejich URI. Proces vytváření komponent vyžaduje několik operací, které na sebe navazují. V případě že by byl tento proces ponechán na vývojáři, tak by byl framework nepoužitelný. Z tohoto důvodu poskytuje třída AFSwinx buildery pro tabulky a formuláře, které komponenty sestaví, vloží do nich data a vývojáři vrátí výsledný JPanel. Typ komponenty určuje vývojář a buildery umí vytvořit formulář či tabulku na základě jedné definice. V případě tabulky je možné ještě provést dodatečné nastavení. Jedná se o automatické nastavení šírky sloupečků a automatické nastavení velikosti tabulky. Ukázka vytvoření formuláře je zobrazena v části zdrojových kódu \ref{code:formGeneration}. Nejprve je potřeba získat instanci builderu, který bude použit. Typ builderu určí zdali bude vytvořena tabulka či formulář. V tomto konkrétním případě bude vytvořen formulář. Metamodel získává klient ze serveru. Framework zapouzdřuje způsob získání dat, vývojář tedy musí definovat zdroje. Jednou z možností je specifikovat zdroje jako samostatné objekty, druhou možností je využít XML. V případě použití XML souboru musí být uveden soubor a identifikátor připojení. Tyto vlastnosti jsou nastaveny builderu pomocí metody initBuilder, která očekává identifikátor formuláře, soubor se specifikací připojení a identifikátor připojení. Builder má samozřejmě několik přetížených metod initBuilder. Tímto lze docílit různých způsobů počátečního nastavení. Metoda buildComponent již postaví vygeneruje výsledný formulář. Pokud se při generování vyskytne chyba, pak je vyhozena vyjímka AFSwinxBuildException. Je na vývojáři jak vyjímku zpracuje. V ukázce je zobrazen dialog s chybovou hláškou.

\begin{lstlisting}[caption={Generování formuláře na klientovi},
  label={code:formGeneration}]
File connectionFile =
	new File(getClass().getClassLoader().getResource("connection.xml").getFile());
	try {
		AFSwinxForm form =
			AFSwinx.getInstance().getFormBuilder()
			.initBuilder("loginForm", connectionFile, "loginFormConnection")
			.buildComponent();
	}catch (AFSwinxBuildException e) {
		getDialogs().failed("afswinx.build.title.failed", "afswinx.build.text.failed", e.getMessage());
	}
\end{lstlisting}

\section{Přenos modelu server klient a generování komponent}
Model na jehož základě jsou generovány komponenty je přenášen ze serveru na klienta. Klient musí tento model správně zpracovat a interpretovat. Nejprve je však potřeba model získat. Pro přenos modelu je použit protokol HTTP či HTTPS. Klientská strana poskytuje vývojáři nativní podporu k získání dat ze serveru. K tomuto účelu je využit framework HttpComponents \cite{apacheHttp}, který poskytuje předpřipravené komponenty, jenž lze využít k vytváření HTTP či HTTPS požadavků. Použitím této komponenty zjednoduší použití našeho frameworku, neboť vývojář nemusí ztrácet čas vytvářením tříd, které by byly schopné získat data ze serveru. Bohužel použití má i nevýhody. Vývojář nemůže ovlivnit implementaci toho, jak jsou data získána, pouze může ovlivnit způsob a to specifikací zdrojů a způsobů připojení. Z tohoto důvodu nabízí framework možnost specifikovat zdroje ve formátu XML s podporou EL. K získání modelu je potřeba mít definovaný zdroj v uzlu metamodel. Uzel se specifikací konkrétních dat a umístění kam data odeslat je nepovinný. Parsování pomocí DOM parseru je však provedeno nad všemi uzly konkrétního připojení a výsledkem je třída AFSwinxConnectionPack, které má reference na konkrétní připojení reprzentovanou třídou AFSwinxConnection. Ukázka je na obrázku \ref{img:connectionPack}. 	Mimo adresy, portu a protokolu lze specifikovat i hlavičku a v případě zabezpečení zdroje autorizaci k tomuto zdroji.

\begin{figure}[h!]
\includegraphics{images/connectionPack}
\caption{Třídy zodpovědné za specifikaci zdrojů a způsobu připojení.}
\label{img:connectionPack}
\end{figure}

\subsection{Generování komponent}
Vývojář na základě builderu určí jaká komponenta se bude generovat. Sekvenční diagram je na obrázku \ref{img:sdDiagram}. Z diagramu je patrné, že builder nejprve získá model ze serveru na základě specifikace zdroje, který mu byl předán během inicializace. Od serveru získá klient třídy reprezentující metamodel. Tento metamodel byl již popsán a je na obrázku \ref{img:metaModelFinal}. Klient nyní začne rekurzivně vytvářet již konkrétní aktivní prvky. Pro každou proměnnou objektu, bude vygenerován widget. Pokud se jedná o neprimitivní datový typ, tak se k příslušnému typu vyhledá jeho reprezentace v metamodelu a generování bude pokračovat tímto objektem. Takovýto přístup zajišťuje zachování pořadí proměnných. Pořadí proměnných lze měnit na serverové straně, nikoliv na klientovi. Typ aktivního prvku určuje atribut widgetType, který je součástí každého popisu konkrétní proměnné. Formulářový builder si nechá vytvořit od tovární třídy WidgetBuilderFactory, která je zodpovědná za vytváření konkrétního builderu, builder jenž je schopný vytvořit požadovaný aktivní prvek. Tento builder vrátí již konkrétní komponenty jako jsou například vstupní textová pole, zaškrtávací políčka a další, před kompletním generováním aktivního prvku lze nastavit jazykové lokalizace a skin. Tyto aktivní komponenty jsou zapouzdřeny v objektu AFSwinxPanel. Důvodem je, že tento panel již zohledňuje layout dané komponenty a mimo aktivní prvek obsahuje i popis, placeholder určený k zobrazení validační hlášky a všechny validace, které musí být nad tímto prvek vykonány. Panel si také udržuje jednoznačný identifikátor v rámci formuláře, na základě kterého lze poté určit jakou proměnnou prvek reprezentuje a její umístění v hiearchii tříd. Panel je následně přidán do dalšího panelu, který udržuje všechny prvky formuláře. V rámci tohoto formulářového panelu jsou také zohledněny layouty a uspořádání komponent. V tomto bodě je formulář sestaven a již nad ním lze provádět validace, či je možné formulář odeslat zpět na server. Nyní je potřeba rozhodnout zdali by měli být ve formuláři zobrazeny data či nikoliv. Formulářový builder vyhodnotí, zdali má naplnit formulář daty a to na základě specifikovaných zdrojů. Pokud byl zdroj s daty specifikován, pak jsou data získána a automaticky vložena do komponenty, jinak se formulář jíž nemodifikuje a práce builderu je ukončena.

\subsubsection{Vkládání dat do komponenty}
Data vkládá do komponenty builder vytváření této komponenty. Komponenta, kterou builder vytváří disponuje funkcionalitou, která ji umožní získat data ze serveru. O datovém objektu, který server poskytuje nemá komponenta předem žádné informace. Proto je tento objekt po obdržení převeden na třídu AFDataPack. Hiearichie je zobrazena na obrázku \ref{img:dataPack}. 
Klíč určuje umístění proměnné v hiearchii. V klíči je použita standardní tečková notace. Například mějme třídu Person, která má referenci na třídu Address přes proměnou myAdress a ve třídě Address je textová proměnná city. Pokud je třída Person první v hiearchii, tak je nahrazena zástupnou hodnotou root. Klíč k proměnné city je pak následující: root.myAdress.city. Stejným způsobem byly vygenerovány klíče pro konkrétní komponenty formuláře, jenž byly sestaveny na základě metadat. Formulář či tabulka mají tedy komponenty, které mají klíče kompatibilní s klíči vygenerovanými z obdržených dat. Lze je tedy spolu spárovat. Problémem však je, že každá z komponent je jiná a byla sestavena specifickým builderem. Nicméně builder zná tento způsob a proto mu byla přidána funkcionalita, na základě které lze upravit současný model a vložit data do již existující komponenty. Typ builderu je určen na základě widgetType. Toto je proměnná, kterou disponují komponenty, jenž byly vytvořeny buildery. Data jsou vkládána do každé komponenty, která byla vytvořena.

\begin{figure}[h!]
\includegraphics{images/dataPack}
\caption{Třídy, na které jsou převedena všechna data, jenž klient obdrží.}
\label{img:dataPack}
\end{figure}

\subsubsection{Widget builder}
V tabulce \ref{table:widgetBuilders} jsou popsány všechny widget buildery, které je možné použít. Všechny buildery mají společného předka. Abstraktní reprezentace buildera je znázorněna na obrázku \ref{img:abstractBuilder}. Konkrétní instance pak využívá společných metod, jenž jsou implementovány v jeho předkovi. Předek umí sestavit placeholder určený k zobrazení výsledku validací, popis ke každé komponentě, nastavit lokalizace, skiny včetně jejich aplikace a přidat validátory na pole. Tato abstraktní třída také umí vygenerovat dummy field, který je vytvořen pouze pokud komponenta dostala list hodnot jež může nabývat a současně není volba z tohoto listu povinná. Pak je zapotřebí udržovat informaci o tom, že uživatel volbu neučinil. Konkrétní prvky builderu z tabulky \ref{table:widgetBuilders} pak vytváří každý builder sám. 

\begin{lstlisting}[caption={Vytváření vstupního pole builderem.},
  label={code:textInputBuilder}]
public AFSwinxPanel buildComponent(AFFieldInfo field) throws IllegalArgumentException, AFSwinxBuildException {
	super.buildBase(field);
	// And input text field
	JTextField textField = new JTextField();
	customizeComponent(textField,field);
	layoutBuilder.addComponent(textField);
	coreComponent = textField;
	// Create panel which holds all necessary informations
	AFSwinxPanel afPanel =
		new AFSwinxPanel(field.getId(), field.getWidgetType(), textField, fieldLabel,
		message);
	// Build layout on that panel
	layoutBuilder.buildLayout(afPanel);
	// Add validations
	super.crateValidators(afPanel, field);
	return afPanel;
    }
\end{lstlisting}

Ukázka metody, která vygeneruje vstupní textové pole znázorněna v části zdrojových kódu \ref{code:textInputBuilder}. Nejprve jsou vytvořeny společné vlastnosti pro všechny buildery. Tyto vlastnosti vytváří abstraktní předek. Poté je na vytvořeno vstupné pole a na toto pole je aplikován skin. Komponenta je poté přidána do layout builderu. Následně je vytvořen AFSwinxPanel, jenž nese všechny nezbytné informace o aktuální komponentě. Tento panel je také přidán do layout builderu, který poté vytvoří konečné úspořádání komponent. Nakonec jsou v panelu registrovány všechny validátory, které se postupně spustí v případě odeslání dat, či v případě žádosti o zjištění validnosti formuláře.

\begin{lstlisting}[caption={Vložení dat do vstupního pole vytvořeného builderem.},
  label={code:textInputBuilderSetData}]
public void setData(AFSwinxPanel panel, AFData data) {
	if (panel.getDataHolder() != null && !panel.getDataHolder().isEmpty()) {
		JTextComponent textField = (JTextComponent) panel.getDataHolder().get(0);
		textField.setText(data.getValue());
	}
}
\end{lstlisting}

Jak již bylo zmíněno, tak builder tím, že zná způsob jakým byly komponenty vytvořeny, tak zná i způsob jakým jsou reprezentovány. V případě potřeby vložení dat do textového pole, je potřeba získat builder, který toto pole vytvořil a požádat ho o vložení dat. Ukázka je v části zdrojových kódů \ref{code:textInputBuilderSetData}. Builder nejprve ověří, zdali existují v panelu komponenty, pokud ano přetypuje je na konkrétní instance, které vytvářel. V tomto případě JTextComponent. Poté jim nastaví data specifickým způsobem pro danou komponentu.


\begin{table}[width=\linewidth]
\begin{center}
\caption{Widget buildery, kterými disponuje klient}
\label{table:widgetBuilders}
\begin{tabular}{|p{4cm}|p{3cm}|p{7cm}|}
\hline
\textbf{Builder} & \textbf{Typ widgetu} & \textbf{Popis} \\
\hline
DateBuilder & 
Calendar & Používá se při reprezentaci datového typu. Umožní uživateli zobrazit date picker, pomocí kterého lze vybrat datum.\\
\hline
DropDownMenuBuilder &
dropDownMenu & Menu, ze kterého lze vybrat jednu z několika voleb.\\
\hline
CheckBoxBuilder & checkBox &
Zaškrtávací políčko, či několik zaškrtávacích políček. Záleží zdali jsou uvedeny možnosti. V případě, že uvedeny nejsou vytvoří se jedno a pokud je zaškrtnuto tak je převedeno na hodnotu true.\\
\hline
InputBuilder & textField &
Builder pro textové pole. Není ničím omezeno.\\
\hline
LabelBuider & label &
Vypíše pouze textovou hodnotu. Do této komponenty nelze vkládat data či ji nijak upravovat. \\
\hline
NumberInputBuilder & numberField	 &
Vytvoří vstupní pole a přidá mu číselnou validaci. \\
\hline
OptionBuilder & option &
Vytvoří skupinu radiobuttonů, z které lze vybrat jednu hodnotu. \\
\hline
PasswordBuilder & password &
Vytvoří vstupní pole, v kterém jsou znaky nahrazeny zástupnými znaky. \\
\hline
TextAreaBuilder & textArea &
Vytvoří vstupní pole pro zadání velkého množství znaků. \\
\hline
\end{tabular}
\end{center}
\end{table}

\subsubsection{Skin}
Widget builder aplikuje na vygenerované komponenty skin. Skin lze nastavit již při získávání formulářového builderu. Skin určuje vzhled konkrétní komponenty. Pomocí skinu lze určit následující vlastnosti.
\begin{enumerate}
\item Barvu, typ fontu, výšku a šířku popisu, který je zobrazen u komponenty.
\item Barvu a typ fontu komponent.
\item Barvu a typ fontu validačních hlášek.
\item Šířku komponent. V případě textových polí i jejich výšku.
\end{enumerate}
Pokud není skin nastaven, tak je použita výchozí implementace, která je součástí frameworku. Vývojář si může definovat vlastní skin a to tak, že buď implementuje rozhranní Skin nebo využije dědičnost a překryje metody z třídy BaseSkin. Výhodou druhého přístupu je fakt, že vývojář může upravit pouze některé metody a nemusí implementovat všechny, které vyžaduje rozhraní.
\section{Přenos a generování dat klient server}
Formuláře a tabulky jsou vytvářeny k tomu, aby reprezentovali uživateli data v systému. Hlavním úkolem formulářů je také odeslání dat na server. Z předchozích sekcí je již zřejmé, že klientská část aplikace nedisponuje stejnými datovými objekty jako server, ale pouze popisem struktury daného objektu. Tato informace je však dostačující a lze na jejím základě vygenerovat data, která je schopný server přijmout. Přenost probíhá v několika krocích. Tyto kroky zachycuje sekvenční diagram na obrázku \ref{img:sdResealization}. Nejprve je zjištěno, zdali bylo při vytváření komponenty specifikován zdroj, na který se mají data odeslat. Před vygenerováním dat, která budou odeslána je provedena validace. V případě, že validace je úspěšná tak se začnou generovat data, která budou odeslána. K tomuto účelu slouží třída JSON builder, v případě že server očekává JSON. Framework nyní podporuje pouze JSON, nicméně návrh počítá s přidáním dalších datových builderů. Tyto buildery již neparsují data, která jsou uloženy v komponentě, za tuto činnost je zodpovědná konkrétní komponenta sama. Komponenta data parsuje z panelů, které si udržuje. Panel má jasně daný klíč, kterým lze určit umístění proměnné v původním objektu. Na základě tohoto klíče je vytvořen nový objekt. Klíč tedy určuje cestu. Pokud je v klíči znak tečky, tak to znamená, že je potřeba vyhledávat v již existující struktuře další potomky. Pokud již v klíči znak tečky není, tak to znamená, že jsme již na správném místě a objektu, který je reprezentován třídou AFDataHolder, jejíž ukázka je na obrázku \ref{img:afDataHolder},  bude přidána do jeho mapy další proměnná s hodnotou. Kromě klíče je potřeba znát i aktuální data v komponentě. Odobně jako při vkládání dat do komponenty je i při získávání dat využit konkrétní widget builder, který komponentu sestavil, neboť zná strukturu a způsob jak data z komponenty získat. JSON builder tedy dostane objekt, z kterého může data sestavit. K sestavení dat je využit framework GSON \cite{gson}. Když jsou již data sestavena tak je postačí odeslat na konkrétní zdroj, který byl specifikován při vytváření komponenty. Framework toto odeslání provede automaticky..

\begin{figure}[h!]
\includegraphics{images/AFDataHolder}
\caption{Třídy, které reprezentuje data ve formuláři. Na jejímž základě je sestaven objekt, který je odeslán na server.}
\label{img:afDataHolder}
\end{figure}

Pokud chce vývojář data odeslat například při kliknutí na tlačítko, tak musí provést pouze dva kroky. Nejprve musí získat konkrétní formulář který chce odeslat a poté nad ním zavolat akci sendData. Formulář lze získat z hlavní třídy AFSwinx na základě jeho identifikátoru kdekoliv v aplikaci. Žádné další akce nejsou potřeba. Mezi hlavní výhody patří snadná použitelnost formulářů a skutečnost, že je vývojář odstíněn od způsobu jakým se data odesílají. Nevýhodou je, že vývojář nemůže plně kontrolovat odeslání dat, či měnit implementaci odesílání a musí používat pouze metody, které mu framework nabízí.
\section{Lokalizace}
\section{Validace dat a vlastní validátory}
\section{Zabezpečení}
\section{Rozdíl mezi komponenty generovanými frameworkem a ručně psanými}


\chapter{Testování}
Testování je nedílnou součástí při vývoji software. Mimo ověření správné funkčnosti aplikace, se také testuje, zdali bylo vytvořeno přesně to, co bylo v zadání. Rozeznáváme automatické a manuální testování. Manuální testování dělají lidé. Výběr těchto testerů by měl odpovídat cílové skupině, pro kterou je aplikace designována. Na toto testování lze nahlížet z několika aspektů. Prvním z nich je zdali aplikace opravdu dělá to co má a druhým aspektem je jak se uživateli s aplikací pracuje. V této kapitole budou popsány různé typy testování, které byly na frameworku provedeny a testovací projekt, který byl v této souvislosti vytvořen. Software disponuje několika Unit testy, které pokrývají chování systému, které provádí složitější generování či parsování. Uživatelský test prověřil použitelnost frameworku a poskytl informace o tom, jak uživatelé pracují s aktuální verzí. Showcase projekt demonstruje možnosti a svým způsobem také testuje software. V tomto případě se jednalo o Alfa test.

\section{Unit testy}
Unit testy testují určitou část software, která by měla být co nejmenší, od toho pochází název unit, neboť se jedná o malou jednotku, jenž test pokrývá. Unit testy pokrývají většinou metody či spolupráci metod a neměli by testovat celkové chování systému. Při návrhu testů je potřeba zvážit, zdali test přinese užitek a zdali může odhalit potencionální chybu. Například testování getterů a setterů je zbytečné. Absolutní pokrytí testů není možné z důvodů časové a prostorové náročnosti, vývoje testů, udržování a možností, které by se museli vygenerovat. Proto je vhodné testovat krajní případy a několik standardních případů použití. 

K testování byl využit framework JUnit \cite{junit}. JUnit testy se spouští při každém buildu aplikace pomocí Mavenu \cite{maven}, pokud není určeno jinak. Framework disponuje spoustou vlastností, které lze použít. Integrace do existujících projektů je jednoduchá a framework podporuje anotace, pomocí kterých lze přizpůsobovat test a určovat chování testy či specifikovat výjimky, které test očekává. Testy mohou být spouštěny ve skupinách. Automatické testování pokrývají následující vlastnosti:
\begin{enumerate}
\item Test správné propagace proměnných do XML souboru s definicemi zdrojů - Do XML souborů je možné vložit hodnoty z hash mapy. Tyto hodnoty mohou za běhu ovlivňovat definici zdrojů a způsob jejich připojení. 
\item Test správné serializace dat získaných z formuláře na JSON - Z formuláře jsou získána data, která mají pouze textovou reprezentaci a na jejich základě je potřeba sestavit JSON. Sestavení JSONu je potřeba ověřit.
\item Test správné vytvoření modelu na základě XML - Model je vytvářen na základě definic. Statické definice jsou součástí zdrojů a testuje se, zdali je na jejich základě možné vygenerovat model. Netestují se tedy definice, ale generátor.
\item Test správného získání možností z určeného výčtového typu - V mnoha případech je potřeba získat data z výčtových typů. V tomto testu se testuje zdali algoritmus, který data získává, funguje korektně.
\end{enumerate}

\section{Uživatelský test}
Návrh frameworku zohledňoval cílovou skupinu uživatelů a snažil se přizpůsobit jejich potřebám. Při návrhu jsem čerpal ze svých vlastních zkušeností, které jsem získal při vývoji software a při své stáži v RedHatu. Firma RedHat vyvíjí v rámci jejich projektu WFK produkt RichFaces \cite{richfaces}. Jedná se o open source komponentový framework pro webové aplikace využívající technologii JSF. Tento projekt má početný tým testerů, který je napojen na upstream. K testování se využívají JUnit testy a Arquillian testy. I přes komplexnost projektu a stabilnost frameworku, bylo rozhodnuto o jeho ukončení. Jedním z důvodů je fakt, že projekt dostatečně rychle nepřinesl podporu mobilních zařízení, dalším důvodem je skutečnost, že na rozdíl od svého přímého konkurenta PrimeFaces \cite{primefaces}, není tak často používaný. Tento fakt lze přikládat použitelnosti framewoku. Při použití komponent je potřeba udělat několik operací, které nemusí být pro začínající uživatele intuitivní. 

\subsection{Cílová skupina}
V mém návrhu jsem se proto soustředil na použitelnost. Cílovou skupinou jsou vývojáři, kteří chtějí knihovnu použít v nové či v stávající aplikaci. Tito vývojáři by měli mít základní zkušenost s webovými službami a v případě využití Swingové části samozřejmě s knihovnou Swing. Architekturu a způsob použití jsem se koncipoval tak aby bylo přímočaré, nicméně bylo potřeba ověřit tuto skutečnost s uživateli během testu. Účastníky testu jsem vybíral ze svých kolegů studentů, o nichž jsem věděl, že patří do cílové skupiny. Také jsem zvolil jako dva z testerů osoby, které již pracují jako softwarový vývojáři. Tímto výběrem jsem chtěl zajistit různorodost informací, které jsem mohl získat během testování. Test jsem provedl se čtyřmi participanty.

\subsection{Test setup}
Vzhledem k tomu, že má framework dvě části, tak bylo potřeba vytvořit prostředí pro serverovou i klientskou část. Serverová část je mnohem komplikovanější, neboť vytvoření aplikace na platformě Java EE s restovým rozhraním a vrstvou jež bude poskytovat data, není triviální věc a samotné nastavení serveru a tvorba tříd, které jsou zodpovědné za tuto funkcionalitu, by zabrala více času než test samotný. Takto koncipovaný test by neměl vypovídající hodnotu o využití testovaného frameworku. Testujícímu byla tedy připravena aplikace, která již tyto vlastnosti splňovala a obsahovala výchozí data. Uvažoval jsem případ, kdy chce vývojář rozšířit aktuální aplikaci o tento framework. Myslím, si že toto bude v praxi nejčastější použití, neboť zatím nepředpokládáme, že by vývojáři tvořili aplikace na míru našemu frameworku. Co se týče klientské strany, tak byla připravena Swingová aplikace, kterou mohl participant upravovat. Klientská aplikace nedisponovala, žádnými knihovny třetích stran. Test probíhal na mém notebooku, který měl předpřipravené prostředí. Tester mohl používat nástroj RestClient pro ověření funkčnosti serverové strany, JBDS vývojové prostředí a byla mu dodána uživatelská příručka. Před testem mu byla vysvětlena základní idea frameworku a byl mu představen framework AspectFaces, který je využit pro generování definic dat. Po testu bylo prostředí resetováno a test se mohl opakovat.

\subsection{Test case}
Při sestavování úkolů pro testera bylo potřeba vzít v úvahu náročnost jednotlivých úkolů. Z tohoto důvodu byly koncipovány úkoly tak, aby na sebe navazovaly. Participantovi byly zadány následující úkoly v níže uvedeném pořadí.
\begin{enumerate}
\item Seznamte se se serverovou a klientskou částí. Prostudujte uživatelskou příručkou.
\item Vytvořte v serverové části zdroj, který bude vracet definici dat objektu Country. Předveďte funkčnost pomocí RestClienta
\item Vytvořte v klientské části prázdný formulář, který bude sestaven na základě definice z předchozího bodu.
\item Naplňte formulář daty ze serveru. Konkrétně zobrazte ve formuláři zemi s identifikačním číslem 1.
\item Vytvořte prázdný formulář stejný jako v bodu 3, který lze odeslat zpět na server a po jeho odeslání bude přidána nová země do databáze.
\item Vytvořte tabulky se všemi zeměmi v databázi a přesvědčte se, zdali byla země s předchozího bodu vložena.
\item Upravte tabulku s předchozího bodu tak, aby byla její velikost závislá na velikosti textů, které zobrazuje.
\item Vytvořte formulář z bodu 3, který bude moci upravovat zemi s identifikačním číslem 1 a po jeho odeslání se změna projeví na serveru v databázi.
\item Vytvořte formulář z bodu 3, který zobrazí zemi s identifikačním číslem 1 a po jeho odeslání bude země ze serverové databáze smazána.
\item Vytvořte formulář z bodu 3. Změňte komponentu, tak aby u proměnné active bylo místo checkboxu dropDownMenu.
\end{enumerate}
Cílem bylo zjistit, zdali jsou uživatelé schopni s frameworkem pracovat, jakým způsobem ho využívají a zdali jsou schopni pracovat s vygenerovanými komponentami.
\subsection{Vyhodnocení}
Všichni uživatelé byli schopni splnit všechny zadané úkoly. V některých případech bylo potřeba testera trochu navést. Tyto případy zde budou rozvedeny. Po skončení testů dostali uživatelé otázku, jak se jim s frameworkem pracovalo. Participanti hodnotili práci kladně, ale měli i pár výtek. Konkrétně jim přišlo, že je framework moc chytrý. Jako příklad zde uvedu bod 3. Při plnění tohoto bodu postupovali podle uživatelské příručky, v které jsou zobrazeny příklady definic zdrojů. Uživatelé vždy zkopírovali zdroj celý a upravili cesty ke zdrojům. To způsobilo, že byl vygenerován formulář, který již obsahoval data. Testeři jako první zkusili vyhledat na formuláři metodu, která data odstraní či se snažili najít metodu, která zakáže zobrazování dat. Až po nějaké době zkusili vymazat definici zdroje. Tento problém se vyskytl u téměř u všech participantů. Výjimkou byl pouze jeden z nich, který nezkopíroval celou definici zdroje, ale jen jeho část. Tento participant však, strávil mnohem větší čas definici zdroje než jeho kolegové.

\begin{table}[width=\linewidth]
\begin{center}
\caption{Problémy se splněním jednotlivých bodů}
\label{table:userTestResult}
\begin{tabular}{|p{7cm}|p{7cm}|}
\hline
\textbf{Popis problému} & \textbf{Návrh řešení} \\
\hline
Pokud je specifikován zdroj, tak jsou zobrazena data ve formuláři, pokud není tak nejsou. Framework si formulář automaticky přizpůsobí. & 
Uživatelé by měli rádi větší kontrolu nad tím, jak se formulář chová. Je vhodné přidat metodu, která zapne či vypne plnění dat do formuláře. \\
\hline
Určení zdroje na základě XML. Problém byl s identifikátorem. Uživatel sice správně identifikátor zvolil, důvodem byl fakt, že ve třídě, která načítá data je identifikátor značen jako connectionKey a v konkrétních XML je reprezentován jako id. & 
Předělat název proměnné ve třídě, tak aby odpovídala XML. \\
\hline
Při vytváření tabulky chtěl uživatel provádět akce navíc. Uživatel nebyl přesvědčený o tom, zdali výměna builderu při zachování zdrojů umožní sestavit tabulku. & Je potřeba upravit uživatelskou příručku, aby tuto informaci přesně specifikovala. \\
\hline
Získání přístupu k již existující komponentě. Uživatel chvilku tápal jak získat komponentu, kterou vygeneroval. Důvodem bylo to, že uživatel chtěl komponentu získávat z panelu do kterého ji vložil. Použití správce komponent AFSwinx mu došlo až po té co si znovu přečetl uživatelskou příručku. & Možná by bylo vhodné zvážit, zdali nepřejmenovat třídu na ComponentManager či použít jiný název, který by více odrážel její funkci. Dále je potřeba upravit uživatelskou příručku a věnovat této komponentě vlastní sekci. \\
\hline
Odesílání dat na severu. Uživatel zde chvilku hledal metodu, jakou lze data odeslat. Problémem bylo to, že výsledná komponenta dědí od JPanel a nabízí tedy velké množství metod. & Řešením by bylo přidat metodu k odeslání do AFSwinx a předat jí formulář, metodu k posílání dat na komponentě je ale potřeba rozhodně zachovat. \\
\hline
\end{tabular}
\end{center}
\end{table}
Uživatelé byly s frameworkem spokojeni. Jeden z uživatelů si poté vytvořil vlastní panel, v kterém zobrazil formulář pro vložení země do databáze a vedle něj umístil tabulku se zeměmi a sledoval, jak se data mění. Další z nich při vytváření tabulky použil již existující kód, který mu vygeneruje formulář a pouze změnil slovo Form na Table a prohlásil, že by to takto mohlo fungovat a program spustil. Tabulka se mu sestavila, ale obsahovala pouze jeden prvek. Změnu zdroje, který získá všechny země, pak uživatel zvládl bez problémů. Test byl pro uživatele i časové náročný neboť se museli seznámit s frameworkem, začít ho správně používat a chybějící metody a chování doprogramovat pomocí frameworku. Během testování bylo sledováno, jak participanti ovládají vygenerované komponenty. S ovládáním komponent neměl žádný participant problém. Vygenerované formuláře a tabulky jim přišli dostatečně velké a přehledné. Při odesílání dat na server byli uživatelé schopni opravit vstupní data, pokud byla validace neúspěšná. 

\section{Ukázkový projekt}
Součástí práce je i poměrně rozsáhlý ukázkový projekt. Tento projekt je rozdělen na dvě části. Serverovou a klientskou část. Obě tyto části demonstrují použití frameworku. Serverová část je odladěna pro aplikační server GlassFish V3 a v4 \cite{glassfish}. Ukázkové projekty se využívají k tomu, aby uživateli ukázali jak framework funguje a v některých případech i k testování, nejčastěji pokud jde vývoj knihoven. Projekt se pak používá k odladění během fáze implementace, poté k testování během fáze testování a k smoke testům před závěrečným releasem. Pokud je ukázkový projekt dostatečně rozsáhlý zabere mnoho času rozšiřování tohoto projektu. V našem případě bylo potřeba implementovat jak serverovou tak klientskou část.
\subsection{Popis projektu}
Ukázkový projekt je zjednodušená aplikace sloužící k zadávání dovolených. Aplikace umožňuje žádat o dovolenou a rušit tyto žádost. Umí spravovat uživatele, také umí vytvářet země, v kterých se uživatel nachází, a ke kterým se váží konkrétní typy pracovní nepřítomnosti. Serverová strana využívá in-memory DerbyDB. Databázová a business vrstva je v ukázkové aplikaci spojená a manažeři, kteří jsou zodpovědní za správu dat, jsou stateless EJB \cite{javaEE}. Webové API je poskytováno pomocí knihovny RestEasy. Zabezpečení realizováno pomocí interceptoru, který na základě basic autorizace a anotacím z javax.annotation.security určí, zdali má uživatel právo k přístupu ke zdroji či nikoliv.

Klientská část aplikace využívá frameworku AFSwinx a získává data ze serverových zdrojů a ty prezentuje uživatelům. Také nabízí vytvoření formulářů a vkládání, mazání či úpravy dat. V aplikaci existuje tzv. security context, který udržuje informace o přihlášeném uživateli a v případě potřeby je odesílá na server. 

\begin{figure}[h!]
\includegraphics[width=\linewidth]{images/Country}
\caption{Ukázkový projekt - klientská část}  
\label{img:country}
\end{figure}

\subsection{Správa zemí}
Na obrázku \ref{img:country} je zobrazen náhled obrazovky zobrazující komponenty potřebné ke správě zemí. Aby si uživatel zobrazil tyto země je potřeba se nejprve přihlásit. Přihlášení je realizováno pomocí metody basic. Obrazovka, na které je zobrazen přihlašovací formulář, je na obrázku \ref{img:loginView}. Aplikace zobrazí všechny země, které jsou k dispozici v zadané v tabulce. Uživatel si může zemi zobrazit v detailním náhledu a popřípadě zemi upravit. Zobrazení detailních informací provede uživatel tak, že klikne na konkrétní zemi a poté na tlačítko choose. Data jsou do formuláře propagována pomocí kontroleru, bez nutnosti stahovat data ze serveru. Tato obrazovka také demonstruje možnost použití zabezpečení. V případě, že je uživatel v roli administrátora, jsou mu umožňěna editace těchto polí. Pokud je uživatel v této roli není, pak jsou políčka needitovatelná. První vrstva zabezpečení je interpretována pomocí klienta a vygenerovaném uživatelském rozhraní. Druhá vrstva je na serveru. V případě obdržení požadavku na změnu se ověří, zdali má uživatel právo měnit zadaná data.
\subsection{Správa typů nepřítomností}
Každá země má svoje vlastní důvody nepřítomnosti a počet dní, které lze na tuto nepřítomnost čerpat. Na obrázku \ref{img:AbsenceType} je zobrazen náhled na správu absenčních typů. Nejprve je potřeba vybrat zemi. Výběrové menu ve vrchní části stránky, je generovaný jednoprvkový formulář,  jehož vnitřní struktura reprezentuje identifikátor konkrétní země a uživateli je zobrazen název dané země. Po výběru konkrétní hodnoty je kontrolerem inicializována akce, při které se začne vytvářet tabulka a formulář, kterým je předán identifikátor vybrané země. Do tabulky jsou naplněna data a je možné je opět měnit pomocí formuláře pod tabulkou.
\subsection{Správa nepřítomností}
Každý uživatel může žádat o schválení nepřítomnosti. Při vytváření zvolí datum začátku, datum konce a typ nepřítomnosti. Uživatel může zadat pouze typ, který odpovídá jeho zemi. Po vytvoření nepřítomnosti se mu tato nepřítomnost zobrazí v přehledu a ve správě nepřítomností ji lze měnit. Správa nepřítomností je na obrázku \ref{img:AbsenceManagementAdmin}. Upravit nepřítomnost lze pouze pokud je ve stavu Requested nebo Cancelled. V případě, že si správu zobrazí správce, tak může měnit nejen své nepřítomnosti, ale také nepřítomnosti ostatních uživatelů. Tuto skutečnost zachycuje obrázek \ref{img:AbsenceManagementAdmin}. Správce má k dispozici také mnohem více stavů. Data do tabulky jsou poskytovány serverem na základě uživatelských rolí. Definice tabulky je pro obě role společná, nicméně definice formuláře je opět závislá na typu role. 

\subsection{Nasazení}
Aplikaci lze bez nutnosti konfigurace nasadit na aplikační server Glassfish verze 3 a 4. K využití lze použít asaadmin, což je nástroj, který umožňuje ovládat aplikační server pomocí příkazové řádky. Soubor s WAR je součástí přiloženého CD či ho lze získat po spuštění mvn clean install v kořenovém adresáři. Poté je potřeba provést následující:
\begin{enumerate}
\item Rozbalte aplikační server GlassFish, který je přiložen na CD nabo si stáhnětě verzi 3 z http://www.oracle.com/technetwork/middleware/glassfish/downloads/java-archive-downloads-glassfish-419424.html Verzi 4 lze stáhnout z \\http://dlc.sun.com.edgesuite.net/glassfish/4.1/release/glassfish-4.1.zip
\item Rozbalte soubor, ve složce bin spusttě utilitu asadmin napsáním asadmin
\item Vložte následující příkaz: start-domain domain1
\item Vložte následující příkaz deploy PATHTOFILE/AFServer.war
\item Jdětě na http://localhost:8080/AFServer - zobrazí se text: I am alive. Serverová strana nedisponuje grafickým uživatelským rozhranním. Funkčnost můžete otestovat rest clienta například na adrese http://localhost:8080/AFServer/rest/country/list - content-type: application/json metoda GET.
\end{enumerate}
Klientskou část aplikace lze spustit s JAR, které je na přiloženém CD či vytvořit spuštěním následujícího příkazu ve složce examples/Showaces. Příkaz je: mvn clean package assembly:single . Do složky target bude vygenerován soubor Showcase.jar, který lze spustit java –jar Showcase.jar . 

\chapter{Závěr}
\section{Budoucí vývoj}
Testování s uživateli dopadlo dobře a na základě toho bychom chtěli framework dále rozvíjet.  Serverová část frameworku umožňuje vytvářet definice komponent. Tyto informace lze využít ke stavbě komponent na libovolné platformě. V další iteraci, bychom rádi přidali možnost využívat tyto definice na mobilní platformě a v dynamicky se rozvíjejícím frameworku AngularJS. S tím se bude pojit vyjmutí specifických modulů z části AFSwinx a jejich propagace do jiného projektu, který pak bude využit na těchto platformách. Z hlediska bezpečnosti lze zatím využít autorizace typu basic, v tomto ohledu bychom rádi projekt rozšířili o další možnosti například o oAuth. Framework nyní podporuje množinu validací, kterou budeme dále rozšiřovat. Budou provedeny další UX testy na základě, kterých může být upraven způsob jakým framework funguje.

Iterace a rozvoj frameworku by měl vyústit v plně nasaditelnou technologii, kterou bychom chtěli distribuovat v rámci AspectFaces. Tím docílíme toho, že vývojář při využítí tohoto projektu definiuje definice pouze jednou a klienti je budou moci na různých platformách používat. Výhodou bude velmi rychlé prototypování a vytváření aktivních prvků, které budou reflektovam změny v serverové konfiguraci a změny v modelu, který reprezentují.
\section{Zhodnocení práce}
Cílem práce bylo navrhout způsob jakým lze aspektově generovat uživatelská rozhranní na platformě Java SE. Práce se zaměřuje především na tlusté klienty k serverovým aplikacím. Framework byl navržen, sestaven a otestován pomocí ukázkové aplikace. Použitelnost frameworku byla testována při testech použitelnosti s cílovou skupinou uživatelů.

Nejprve bylo potřeba zpracovat již existující řešení a zhodnotit jejich výhody a nevýhod. Již v této fázy bylo zřejmé, že pokud budou objekty dat centralizovány, tak bude zjednodušena jejich správa a možné úpravy. Způsob má i jednu nevýhodu, pokud si klient od serveru vyžádá data, tak je struktura, kterou obdrží závislá na datech. V případě že se mění struktura dat, musín se přizpůsobovat i klient. Tuto nevýhodu lze odstranit dynamickou inspekcí dat, která vytvoří definice a aspektový přístup, jenž nejprve využije defince k sestavení komponenty a poté ji nastaví na základě obdržených dat. Toto nastavení je dynamické, klient se tedy nemusí přizpůsobovat pokud server změní strukturu dat, neboť mu to server sdělí ihned při generování definic.

Byl vytvořen návrh definic dat, pomocí kterých umí klient vytvořit formuláře a tabulky. Tyto definice jsou generovány na serverové straně s využitím frameworku AspectFaces a poté odeslány na klienta, kde jsou interpretovány. Klient umí na základě těchto definic vytvářet tabulky či formuláře, které umí sestavit podle určeného layoutu a nastavit všem aktivním prvkům validátory. Při obdržení dat umí tyto data správně vložit do formuláře a následně datový objekt znovu sestavit pouze ze znalosti definice. Klientská strana předem nezná objekt, jenž obdrží a může tedy pružně reagovat na změny datových definic. Z hlediska bezpečnosti podporuje klient komunikaci pomocí HTTPS a autorizaci typu basic. Framework, je tedy rozdělen na klientskou a serverou část s tím, že klintovi stačí vložení závislosti na klientskou stranu a serveru na servervou stranu a v případě využití AspectFaces i závislost na tento projekt.

Ke generování dat se využívá knihovna třetí strany. Výhodou je její stabilnost a fakt, že disponuje velkou škálou možností, jak uživatelské rozhranní generovat. Framework zohledňuje uživatelské role a nabízí anotace, které lze využít při generování. Šablony byly upraveny tak, aby vytvářeli jednotnou definice použitelnou na více platformách. Definice je tedy platformově nezávislá. 






%*****************************************************************************
% Seznam literatury je v samostatnem souboru reference.bib. Ten
% upravte dle vlastnich potreb, potom zpracujte (a do textu
% zapracujte) pomoci prikazu bibtex a nasledne pdflatex (nebo
% latex). Druhy z nich alespon 2x, aby se poresily odkazy.

% originally following specification for bibliography formating was used
%\bibliographystyle{abbrv}

% Here is an improvment by Petr Dlouhy (April 2010).
% It is mainly for supervisors who expect Czech fomrating rules for references
% Additional feature is live url addresses to sources from your pdf file
% It requires the file csplainnat.bst (included in this sample zipfile).
\bibliographystyle{csplainnat}

%bibliographystyle{plain}
%\bibliographystyle{psc}
{
%JZ: 11.12.2008 Kdo chce mit v techto ukazkovych odkazech take odkaz na CSTeX:
\def\CS{$\cal C\kern-0.1667em\lower.5ex\hbox{$\cal S$}\kern-0.075em $}
\bibliography{reference}
}

\appendix
\chapter{Seznam použitých zkratek}

\begin{description}
\item[GNU GPL] GNU General Public License
\item[UML] Unified Modeling Language
\item[MVC] Model-view-controller
\item[REST] Representational State Transfer
\item[JDBC] Java Database Connectivity
\item[OSGI] Open Services Gateway Initiative
\item[HTTP] Hypertext Transfer Protocol
\item[ORM] Object relational mapping
\item[EJB] Enterprise JavaBean
\item[JPA] Java Persistence API
\item[API] Application programming interface
\item[EL] Expression language
\item[HTML] HyperText Markup Language
\item[XML] Extensible Markup Language
\item[JSF] JavaServer Faces
\item[LGPL EPL] Lesser GPL Eclipse Public Licence
\item[SSL] Secure Sockets Layer
\item[HTTPS] Hypertext Transfer Protocol Secure
\item[WAR] Web Application Archive
\item[JAR] Java Archive
\end{description}

%*****************************************************************************
\chapter{Instalační a uživatelská příručka}
Framework byl vytvořen jako Maven projekt, jeho přidání do již existující aplikace lze buďto jako knihovnu nebo jako Maven závislost. Způsob jakým lze integrovat projekt a jak ho používat je detailně popsán v uživatelské příručce na přiloženém CD. 
\section{Maven závislosti}
Nejprve je potřeba provést build frameworku. Zdorojové kódy jsou na přiloženém CD. Framework zatím není v žádném z veřejně dostupných repozitářích. Poté lze na serverou stranu přidat následující závislosti:
\begin{lstlisting}[caption={Závislosti na serveru},
  label={code:mavenDependency}]
<dependency>
	<groupId>com.tomscz.af</groupId>
	<artifactId>AFRest</artifactId>
	<version>0.0.1-SNAPSHOT</version>
</dependency>
<dependency>
	<groupId>com.codingcrayons.aspectfaces</groupId>
	<artifactId>javaee-connector</artifactId>
	<version>1.5.0-SNAPSHOT</version>
</dependency>
<dependency>
	<groupId>com.codingcrayons.aspectfaces</groupId>
	<artifactId>annotation-descriptors</artifactId>
	<version>1.5.0-SNAPSHOT</version>
</dependency>
\end{lstlisting}
Repozitář pro aspectFaces je zde:
\begin{lstlisting}[caption={Aspect faces repozitář	},
  label={code:mavenAspectFacesRepo]
<repository>
	<id>codingcrayons-repository</id>
	<name>CodingCrayons Maven Repository</name>
	<url>http://maven.codingcrayons.com/content/groups/public/</url>
</repository>
\end{lstlisting}
Do složky WEB-INF je potřeba rozbalit soubor templates.zip, v kterém je předpřipravená konfigurace a do web.xml je potřeba přidat listener, který provede nastavení AspectFaces během startu. 
\begin{lstlisting}[caption={Aspect faces repozitář	},
  label={code:mavenAspectFacesBootStrap]
<listener>
	<!-- Include Aspect Faces listener to perform proper framework initialization 
		during application start -->
	<listener-class>com.codingcrayons.aspectfaces.plugins.j2ee.AspectFacesListener</listener-class>
</listener>
\end{lstlisting}
Na klientskou stranu je potřeba přidat následující závistlost:
\begin{lstlisting}[caption={Závislost na klientské straně},
  label={code:mavenAFSwinx]
<dependency>
	<groupId>com.tomscz.af</groupId>
	<artifactId>AFSwinx</artifactId>
	<version>0.0.1-SNAPSHOT</version>
</dependency>
\end{lstlisting}
\section{Ukázkový projekt}
Ukázkový projekt se svojí klientskou a serverou částí jsou již vytvořeny a přiloženy na CD. Serverovou část lze bez dodatečné konfigurace spusti na serveru GlassFish V4. Je potřeba provést následující:
\begin{enumerate}
\item Stáhněte si aplikační server GlassFish http://dlc.sun.com.edgesuite.net/glassfish/4.1/release/glassfish-4.1.zip
\item Rozbalte soubor a v bin složce spusttě utilitu asadmin
\item Vložte následující příkaz: start-domain domain1
\item Vložte následující příkaz deploy CESTA_K_SOUBORU/AFServer.war
\item Jdětě na http://localhost:8080/AFServer - zobrazí se text: It alive. Serverová strana nedisponuje grafickým uživatelským rozhranním. Funkčnost můžete otestovat rest clienta například na adrese http://localhost:8080/AFServer/rest/country/list - content-type: application/json metoda GET.
\end{enumerate}
Nyní je potřeba spustit klientskou část aplikace. Ve složce s Showcase.jar, který je přiložen na CD spusťte java -jar Showcase.jar . Aplikace bude spuštěna.



\chapter{UML diagramy}
V této sekci naleznete použité UML diagramy, na které bylo v textu odkazováno.
\begin{figure}
\begin{center}
\includegraphics[angle=270]{images/useCase}
\caption{Případy užití frameworku}
\label{img:useCase}
\end{center}
\end{figure}

\begin{figure}
\includegraphics{images/businessModel}
\caption{Business model}
\label{img:businessModel}
\end{figure}

\begin{figure}
\begin{center}
\includegraphics{images/deploymentDiagram}
\caption{Diagram nasazení}
\label{img:deploymentFrameworkDiagram}
\end{center}
\end{figure}

\begin{figure}
\begin{center}
\includegraphics{images/serverSide}
\caption{Diagram balíčků a jejich tříd z AFRest}
\label{img:serverSide}
\end{center}
\end{figure}

\begin{figure}
\includegraphics[angle=270]{images/metaModelFinal}
\caption{Doménový model obecných definic komponent}
\label{img:finalModel}
\end{figure}

\begin{figure}
\includegraphics[angle=270]{images/sdDiagram}
\caption{SD diagram sestavení formuláře}
\label{img:sdDiagram}
\end{figure}	

\begin{figure}
\includegraphics[angle=270]{images/abstractBuilder}
\caption{Doménový model, znázorňující buildery, které jsou použity při vytváření aktivních prvků}
\label{img:abstractBuilder}
\end{figure}	

\begin{figure}
\includegraphics[angle=270]{images/sdResealization}
\caption{SD diagram odeslání dat na server}
\label{img:sdResealization}
\end{figure}	

\chapter{Ukázky zdrojového kódu a XML souborů}
V této sekci naleznete ukázky zdrojového kódu, na které bylo v textu odkazováno.
\begin{lstlisting}[caption=Ukázka definice komponenty,
  label={code:xmlCompleteDefinition}]
<?xml version="1.0" encoding="UTF-8" standalone="no"?>
<afRestEntity xmlns:xsi="http://www.w3.org/2001/XMLSchema-instance">
	<entity>
		<entityName>person</entityName>
		<entity>
			<entityName>address</entityName>
			<widget>
				<widgetType />
				<fieldName>street</fieldName>
				<label>Street</label>
				<validations>
					<required />
				</validations>
				<fieldLayout>
					<layoutOrientation />
					<labelPossition />
					<layout />
				</fieldLayout>
			</widget>
			<fieldName>myAdress</fieldName>
		</entity>
		<widget>
			<widgetType />
			<fieldName>firstName</fieldName>
			<label>person.firstName</label>
			<validations>
				<required>true</required>
			</validations>
			<fieldLayout>
				<layoutOrientation />
				<labelPossition />
				<layout />
			</fieldLayout>
		</widget>
		<widget>
			<widgetType>textArea</widgetType>
			<fieldName>lastName</fieldName>
			<label>person.lastName</label>
			<validations>
				<required />
			</validations>
			<fieldLayout>
				<layoutOrientation>AxisX</layoutOrientation>
				<labelPossition>before</labelPossition>
				<layout>TwoColumnsLayout</layout>
			</fieldLayout>
		</widget>
		<widget>
			<widgetType>checkBox</widgetType>
			<fieldName>confidentialAgreement</fieldName>
			<label>Confidential Agreement</label>
			<validations>
				<required />
			</validations>
			<fieldLayout>
				<layoutOrientation />
				<labelPossition />
				<layout />
			</fieldLayout>
		</widget>
	</entity>
</afRestEntity>
\end{lstlisting}



%*****************************************************************************

%*****************************************************************************
\chapter{Obsah přiloženého CD}

\begin{figure}[h!]
\begin{center}
\includegraphics{images/cdContent}
\caption{Obsah přiloženého CD}  
\label{img:cd}
\end{center}
\end{figure}
\end{document}
